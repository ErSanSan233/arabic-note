\chapter{金箍棒重多少斤}

\section{\lecon{14.1} 认识数词}


阿拉伯印度数字的书写方向与阿拉伯数字完全相同。例如 2025 直接写作 \arm{٢٠٢٥}。

各数字名称如下:

\begin{Arabic}
    \begin{multicols}{3}
    \begin{description}
        \item[٠] صِفْرٌ
        \item[١] وَحِدٌ
        \item[٢] اِثْنَانِ
        \item[٣] ثَلاثَةٌ
        \item[٤] أَرْبَعَةٌ
        \item[٥] خَمْسَةٌ
        \item[٦] سِتَّةٌ
        \item[٧] سَبْعَةٌ
        \item[٨] ثَمَانِيَةٌ
        \item[٩] تِسْعَةٌ
        \item[١٠] عَشَرَةٌ
        \item[~] 
    \end{description}
    \end{multicols}
\end{Arabic}

〔法〕zéro/chiffre、〔英〕cipher等词与\arm{صِفْرٌ} ( \arm{٠} )同源。

\newfontfamily\handwritingarabicfont[
    Script=Arabic,
    ItalicFont=Mishafi,
    Scale=1.6 
]{Mishafi}

\begin{attention}
    \arm{٢} 可以写为 \handwritingarabicfont{٢}。
\end{attention}

\begin{note}
    此外,课上强调 \arm{١} 手写时需要倾斜,似乎倾斜是一项不能轻易舍弃的重要特征。 \arm{٢} 的另一种形态 \handwritingarabicfont{٢} 本质上是上端可以忽略向下的弧度,在艺术字中往往会呈现为横平竖直的折线,与 \arm{٦} 完全对称呈现。
\end{note}

\section{\lecon{14.2} \lecon{14.3} 基数词1~10}

\begin{note}
    本课从此处开始,我觉得课上讲的规则有点过于复杂了。但考虑到这几节课的大多数资源都没有声音,这里先尽量多抄录,之后再整合。
\end{note}

1、2位于被数名词后面,性、格、式需要配合。

\begin{Arabic}
    \begin{center}
        \begin{tabular}{c|ccc}
        \crm{数字} & \crm{主} & \crm{宾} & \crm{属} \\
        \hline
        ١ & وَاحِدٌ م وَحِدَةٌ& وَاحِدًا م وَحِدَةً & وَاحِدٍ م وَحِدَةٍ \\
        ٢ & اِثْنَانِ م اِشْنَتَانِ & \multicolumn{2}{c}{اِثْنَيْنِ م اِثْنَتَيْنِ} \\
    \end{tabular}
    \end{center}
\end{Arabic}

3~10位于被数名词前面,标单音符,性与被数名词\emph{相反}。此时,被数名词只能是复数属格泛指式,与基数词构成正偏组合。

以下是主格接 \arm{قَلَمٌ جـ أَقْلَامٌ} (阳性)及 \arm{مِقْلَمَةٌ جـ مَقَالِمُ} (阴性) 的例子:

\begin{attention}
    \arm{مَقَالِةُ} 词尾的 \arm{ـُ} 表示该词为半变尾名词,即其宾属格同形标 \arm{ـَ}。
\end{attention}

\begin{note}
    目前暂时不清楚词尾的 \arm{ـُ} 是否标志着该词为半变尾名词。
\end{note}

\begin{Arabic}
    \begin{center}
        \begin{tabular}{c|c|ccl}
            \crm{数字} & \crm{本名} & قَلَمٌ & مِقْلَمَةٌ & \\
            \hline
            ١ & وَحِدٌ & \gray{قَلَمٌ} وَاحِدٌ & \gray{مِقْلَمَةٌ} وَاحِدَةٌ \\
            ٢ & اِثْنَانِ & \gray{قَلَمَانِ} اِثْنَانِ & \gray{مِقْلَمَتَانِ} اِثْنَتَانِ \\
            ٣ & ثَلاثَةٌ & ثَلَاثَةُ\red{$^*$} \gray{أَقْلَامٍ} & ثَلَاثُ \gray{مَقَالِمَ} \\
            ٤ & أَرْبَعَةٌ & أَرْبَعَةُ \gray{أَقْلَامٍ} & أَرْبَعُ \gray{مَقَالِمَ} \\
            ٥ & خَمْسَةٌ & خَمْسَةُ \gray{أَقْلَامٍ} & خَمْسُ \gray{مَقَالِمَ} \\
            ٦ & سِتَّةٌ & سِتَّةُ \gray{أَقْلَامٍ} & سِتُّ \gray{مَقَالِمَ} \\
            ٧ & سَبْعَةٌ & سَبْعَةُ \gray{أَقْلَامٍ} & سَبْعُ \gray{مَقَالِمَ} \\
            ٨ & ثَمَانِيَةٌ & ثَمَانِيَةٌ \gray{أَقْلَامٍ} & ثَمَانِي\red{$^\dagger$} \gray{مَقَالِمَ} \\
            ٩ & تِسْعَةٌ & تِسْعَةُ \gray{أَقْلَامٍ} & تِسْعُ \gray{مَقَالِمَ} \\
            ١٠ & عَشَرَةٌ & عَشَرَةُ \gray{أَقْلَامٍ} & عَشْرُ\red{$^\ddagger$} \gray{مَقَالِمَ} \\
        \end{tabular}
    \end{center}
\end{Arabic}

\begin{footnotesize}
\begin{itemize}
    \item [\red{$^*$}] 注意单音符,下同。
    \item [\red{$^\dagger$}] \arm{٨} 的阳性 \arm{شَمَانٍ} 为缺尾名词,词尾不表示格位;改单音符时,改为长音(即所谓``写出暗藏的 \arm{ى} ,把鼻音符打开'')。
    \item [\red{$^\ddagger$}] \arm{١٠} 的阳性 \arm{عَشْرٌ} 拼写特殊,注意不是 \arm{ـشَـ} 。
\end{itemize}
\end{footnotesize}

\begin{note}
    前面提了好几次``阳性指物名词复数当成阴性单数看待'',这里其实数字 3~10 恰好在某种程度上呼应了这样的规则。我觉得`` 3~9 反阴阳''实际是性的搭配``负负得正''的结果:本来数词就已经用阴性表示复数了,再配合被数名词的阴性,反而呈现成阳性的态势。

    此外,\arm{اِثْنَانِ} 本身也满足名词双数的变位规则。
\end{note}

\section{\lecon{14.4} 基数词11~99}

位于被数名词之前,个位数 1~9 性依前。被数名词均为单数宾格泛指式,作基数词的\emph{区分语}。

\subsection{11~19}

个位在前,十位在后(拼写特殊),二者词尾都标 \arm{ـَ};个位性变化依前,十位性与被数名词一致;仅 12 有格的变化。形如:

\begin{center}
    \arm{\crm{个位}ــَ + \crm{十位}ــَ + \xcancel{الـ}ـ\crm{名词}ــًا}
\end{center}

\begin{Arabic}
    \begin{center}
        \begin{tabular}{c|cc}
            \crm{数字} & قَلَمٌ & مِقْلَمَةٌ \\
            \hline
            ١١ & أَحَدَ عَشَرَ \gray{قَلَمًا}& إِحْدَى عَشْرَةَ\red{$^*$} \gray{مِقْلَمَةً}\\
            ١٢\red{$^\dagger$} & اِثْنَا عَشَرَ \gray{قَلَمًا}& اِثْنَتَا عَشْرَةَ \gray{مِقْلَمَةً}\\
            ١٣ & ثَلَاثَةَ عَشَرَ\red{$^\ddagger$} \gray{قَلَمًا}& ثَلَاثَ عَشْرَةَ \gray{مِقْلَمَةً}\\
            ١٤ & أَرْبَعَةَ عَشَرَ \gray{قَلَمًا} & أَرْبَعَ عَشْرَةَ \gray{مِقْلَمَةً}\\
            ١٥ & خَمْسَةَ عَشَرَ \gray{قَلَمًا} & خَمْسَ عَشْرَةَ \gray{مِقْلَمَةً}\\
            ١٦ & سِتَّةَ عَشَرَ \gray{قَلَمًا} & سِتَّ عَشْرَةَ \gray{مِقْلَمَةً}\\
            ١٧ & سَبْعَةَ عَشَرَ \gray{قَلَمًا} & سَبْعَ عَشْرَةَ \gray{مِقْلَمَةً}\\
            ١٨ & ثَمَانِيَةَ عَشَرَ \gray{قَلَمًا} & ثَمَانِيَ عَشْرَةَ \gray{مِقْلَمَةً}\\
            ١٩ & تِسْعَةَ عَشَرَ \gray{قَلَمًا} & تِسْعَ عَشْرَةَ \gray{مِقْلَمَةً}\\
        \end{tabular}
    \end{center}
\end{Arabic}

\begin{footnotesize}
\begin{itemize}
    \item [\red{$^*$}] 注意 10 作十位数时阴阳性跟词中的 \arm{ـشـ} 发音符号搭配( \arm{ـشَـٌ م ـشْـةٌ} )跟单独用时 ( \arm{ـشْـٌ م ـشَـةٌ} ) 相反。
    \begin{note}
        其实搭配上被数名词之后就没那么混乱了,所以这里就是搭阳性名词读 \arm{ـشَـ} ,搭阴性名词读 \arm{ـشْـ} ,别管数字现在是阴性还是阳性。
    \end{note}
    \item [\red{$^\dagger$}] 有变格,展示的是主格,宾属格为 \arm{اِثْنَيْ عَشَرَ \gray{(قَلَمًا)} م اِثْنَتَيْ عَشْرَةَ \gray{(مِقْلَمَةً)}} 。同时,2 作 12 的个位数时,各格都去掉了词尾 \arm{نِ} 。
    \item [\red{$^\ddagger$}] \arm{٣} 为阴性,\arm{١} 为阳性,此时的 \arm{١٣} 整体上称作``阴性的 13''。
\end{itemize}


\end{footnotesize}

\subsection{10$\mathbb{N} $}

位于被数名词前,不变性,只变格,同名词完整式阳性复数词尾变格相同。

\begin{Arabic}
    \begin{center}
        \begin{tabular}{c|cc}
            \crm{数字} & \crm{主格} & \crm{宾、属格} \\
            \hline
            ٢٠ & عِشْرُونَ \gray{قَلَمًا/مِقْلَمَةً}& عِشْرِينَ \gray{قَلَمًا/مِقْلَمَةً}\\
            ٣٠ & ثَلاَثُونَ \gray{قَلَمًا/مِقْلَمَةً}& ثَلاَثِينَ \gray{قَلَمًا/مِقْلَمَةً}\\
            ٤٠ & أَرْبَعُونَ \gray{قَلَمًا/مِقْلَمَةً}& أَرْبَعِينَ \gray{قَلَمًا/مِقْلَمَةً}\\
            ٥٠ & خَمْسُونَ \gray{قَلَمًا/مِقْلَمَةً}& خَمْسِينَ \gray{قَلَمًا/مِقْلَمَةً}\\
            ٦٠ & سِتُّونَ \gray{قَلَمًا/مِقْلَمَةً}& سِتِّينَ \gray{قَلَمًا/مِقْلَمَةً}\\
            ٧٠ & سَبْعُونَ \gray{قَلَمًا/مِقْلَمَةً}& سَبْعِينَ \gray{قَلَمًا/مِقْلَمَةً}\\
            ٨٠ & ثَمَانُونَ \gray{قَلَمًا/مِقْلَمَةً}& ثَمَانِينَ \gray{قَلَمًا/مِقْلَمَةً}\\
            ٩٠ & تِسْعُونَ \gray{قَلَمًا/مِقْلَمَةً}& تِسْعِينَ \gray{قَلَمًا/مِقْلَمَةً}\\
        \end{tabular}
    \end{center}
\end{Arabic}

\subsection{其余}

\begin{center}
    \arm{\crm{个位} + وَ\crm{十位} + \xcancel{الـ}ـ\crm{名词}ــًا}
\end{center}

十位不变性,个位变性依前,均变格。

\begin{Arabic}
    \begin{center}
        \begin{tabular}{c|cc}
            \crm{数字} & قَلَمٌ & مِقْلَمَةٌ \\
            \hline
            ٢١ & وَاحِدٌ وَعِشْرُونَ \gray{قَلَمًا}&  وَاحِدَةٌ وَعِشْرُونَ \gray{مِقْلَمَةً}\\
            ٢٢ & اِثْنَانِ وَعِشْرُونَ \gray{قَلَمًا}& اِثْنَتَانِ وَعِشْرُونَ \gray{مِقْلَمَةً}\\
            ٣٣ & ثَلاثَةٌ وَثَلاَثُونَ \gray{قَلَمًا}& ثَلاثٌ وَثَلاَثُونَ \gray{مِقْلَمَةً}\\
            ٤٤ & أَرْبَعَةٌ وَأَرْبَعُونَ \gray{قَلَمًا}& أَرْبَعٌ وَأَرْبَعُونَ \gray{مِقْلَمَةً}\\
            ٥٥ & خَمْسَةٌ وَخَمْسُونَ \gray{قَلَمًا}& خَمْسٌ وَخَمْسُونَ \gray{مِقْلَمَةً}\\
            ٦٦ & سِتَّةٌ وَسِتُّونَ \gray{قَلَمًا}& سِتٌّ وَسِتُّونَ \gray{مِقْلَمَةً}\\
            ٧٧ & سَبْعَةٌ وَسَبْعُونَ \gray{قَلَمًا}& سَبْعٌ وَسَبْعُونَ \gray{مِقْلَمَةً}\\
            ٨٨ & ثَمَانِيَةٌ وَثَمَانُونَ \gray{قَلَمًا}& ثَمَانٍ وَثَمَانُونَ \gray{مِقْلَمَةً}\\
            ٩٩ & تِسْعَةٌ وَتِسْعُونَ \gray{قَلَمًا}& تِسْعٌ وَتِسْعُونَ \gray{مِقْلَمَةً}\\
        \end{tabular}
    \end{center}
\end{Arabic}

变格时,记得个位和十位同时变格:

\begin{Arabic}
    \begin{center}
        \begin{tabular}{c|cc}
            ٣١ & قَلَمٌ & مِقْلَمَةٌ \\
            \hline
            \crm{主} & وَاحِدٌ وَثَلاَثُونَ \gray{قَلَمًا} & وَاحِدَةٌ وَثَلاَثُونَ \gray{مِقْلَمَةً}\\
            \crm{宾} & وَاحِدًا وَثَلاَثِينَ \gray{قَلَمًا} & وَاحِدَةً وَثَلاَثِينَ \gray{مِقْلَمَةً}\\
            \crm{属} & وَاحِدٍ وَثَلاَثِينَ \gray{قَلَمًا} & وَاحِدَةٍ وَثَلاَثِينَ \gray{مِقْلَمَةً}\\
            \hline
            \hline
            ٣٥ \\
            \hline
            \crm{主} & خَمْسَةٌ وَثَلاَثُونَ \gray{قَلَمًا} & خَمْسٌ وَثَلاَثُونَ \gray{مِقْلَمَةً}\\
            \crm{宾} & خَمْسَةً وَثَلاَثِينَ \gray{قَلَمًا} & خَمْسًا وَثَلاَثِينَ \gray{مِقْلَمَةً}\\
            \crm{属} & خَمْسَةٍ وَثَلاَثِينَ \gray{قَلَمًا} & خَمْسٍ وَثَلاَثِينَ \gray{مِقْلَمَةً}\\
        \end{tabular}
    \end{center}
\end{Arabic}

\section{\lecon{14.5} 基数词百}

$100\mathbb{N} $整百时,位于被数名词之前,词尾标单音符。不变性。被数名词均为单数属格泛指式(与基数词构成正偏组合)。

百带有十位和个位时,位于被数名词之前,依次说百、个、十位。个位变性依前。被数名词均为单数宾格泛指式(作基数词的区分语)。

自 300 起,不表示为``数字+百'',而是有一套特殊的\emph{复合名词},词尾标齐齿符。如果单独作为数字,为 \arm{ـٍ} ;若接被数名词,为 \arm{ـِ} 。

\begin{Arabic}
    \begin{multicols}{2}
        \begin{description}
            \item[١٠٠] مِائَةٌ جـ مِائَاتٌ 
            \item[٢٠٠] مِائَتَانِ
            \item[٣٠٠] ثَلَاثُمِائَةٍ
            \item[٤٠٠] أَرْبَعُمِائَةٍ
            \item[٥٠٠] خَمْسُمِائَةٍ
            \item[٦٠٠] سِتُّمِائَةٍ
            \item[٧٠٠] سَبْعُمِائَةٍ
            \item[٨٠٠] ثَمَانِمِائَةٍ
            \item[٩٠٠] تِسْعُمِائَةٍ 
            \item[~] 
        \end{description}
    \end{multicols}
\end{Arabic}

\begin{attention}
    \arm{ـمِاـ} 均读 \arm{ـمِـ} 。

    \arm{مِائَاتٌ} 只用于抽象上的若干百,不会出现在具体数字中。
    
    \arm{مِائَتَانِ} 即 \arm{مِائَةٌ} 的双数形式,宾属格依规则变为 \arm{مِائَتَيْنِ}。

    自 \arm{٣٠٠} 起,除 \arm{٨٠٠} 不体现变格外,各数字格位变化均由 \arm{ـمِائَةٍ}(\arm{ـمِائَةِ})前的发音符号体现。即:主格 \arm{ـُمِائَةٍ}(\arm{ـُمِائَةِ}) ,宾格 \arm{ـَمِائَةٍ}(\arm{ـَمِائَةِ}),属格 \arm{ـِمِائَةٍ}(\arm{ـِمِائَةِ})。
\end{attention}

\begin{Arabic}
    \begin{center}
        \begin{tabular}{c|cc}
            \crm{数字} & قَلَمٌ & مِقْلَمَةٌ \\
            \hline
            ١٠٠ & مِائَةُ \gray{قَلَمٍ}& مِائَةُ \gray{مِقْلَمَةٍ}\\
            ١٠١ & مِائَةُ \gray{قَلَمٍ} وَقَلَمٌ\red{$^*$} \\
            ١٠٢ & & مِائَةُ \gray{مِقْلَمَةٍ} وَمِقْلَمَتَانِ\red{$^*$}\\
            ٢٠٠\red{$^\dagger$} & مِائَتَا\red{$^\ddagger$} \gray{قَلَمٍ} & مِائَتَا\red{$^\ddagger$} \gray{مِقْلَمَةٍ}\\
            ٥٠٠ & خَمْسُمِائَةِ \gray{قَلَمٍ} & خَمْسُمِائَةِ \gray{مِقْلَمَةٍ}\\
            ٥٢١ & خَمْسُمِائَةٍ وَوَحِدٌ وَعِشْرُونَ \gray{قَلَمًا} & خَمْسُمِائَةٍ وَوَحِدَةٌ وَعِشْرُونَ \gray{مِقْلَمَةً}\\
            ٥٢٥ & خَمْسُمِائَةٍ وَخَمْسَةٌ وَعِشْرُونَ \gray{قَلَمًا} & خَمْسُمِائَةٍ وَخَمْسٌ وَعِشْرُونَ \gray{مِقْلَمَةً}\\
        \end{tabular}
    \end{center}
\end{Arabic}

\begin{footnotesize}
    \begin{itemize}
        \item [\red{$^*$}] 注意101和102的特殊表达。形如``101支笔''的数字应表达为``100笔和笔''。
        \item [\red{$^\dagger$}] 宾属格为 \arm{مِائَتَيْ قَلَمٍ / مِائَتَيْ مِقْلَمَةٍ}。
        \item [\red{$^\ddagger$}] 注意要去掉 \arm{نِ} (宾属格也要去掉),没说为啥。
    \end{itemize}
\end{footnotesize}

变格时,三位数字均变格。举例:

\begin{Arabic}
    \begin{center}
        \begin{tabular}{c|cc}
            ٥٢٥ & قَلَمٌ & مِقْلَمَةٌ \\
            \hline
            \crm{主} & خَمْسُمِائَةٍ وَخَمْسَةٌ وَعِشْرُونَ \gray{قَلَمًا} & خَمْسُمِائَةٍ وَخَمْسٌ وَعِشْرُونَ \gray{مِقْلَمَةً}\\
            \crm{宾} & خَمْسَمِائَةٍ وَخَمْسَةً وَعِشْرِينَ \gray{قَلَمًا} & خَمْسَمِائَةٍ وَخَمْسًا وَعِشْرِينَ \gray{مِقْلَمَةً}\\
            \crm{属} & خَمْسِمِائَةٍ وَخَمْسَةٍ وَعِشْرِينَ \gray{قَلَمًا} & خَمْسِمِائَةٍ وَخَمْسٍ وَعِشْرِينَ \gray{مِقْلَمَةً}\\
        \end{tabular}
    \end{center}
\end{Arabic}

\section{\lecon{14.6} 基数词$10^3, 10^6, 10^9, \cdots$}

对于大数,阿语每三位分组命名。

\begin{note}
    感觉老师这一节已经沉浸在自己的艺术世界里了……
\end{note}

\subsection{千}

不带个位、十位时,依整百规则(正偏组合);带个位、十位时,从大到小说,到百后依百规则(区分语)。

\begin{Arabic}
    \begin{multicols}{2}
        \begin{description}
            \item[١٠٠٠] أَلْفٌ جـ آلَافٌ
            \item[٢٠٠٠] أَلْفَانِ
            \item[٣٠٠٠] ثَلَاثَةُ آلَافٍ
            \item[٤٠٠٠] أَرْبَعَةُ آلَافٍ
            \item[٥٠٠٠] خَمْسَةُ آلَافٍ
            \item[٦٠٠٠] سِتَّةُ آلَافٍ
            \item[٧٠٠٠] سَبْعَةُ آلَافٍ
            \item[٨٠٠٠] ثَمَانِيَةُ آلَافٍ
            \item[٩٠٠٠] تِسْعَةُ آلَافٍ    
            \item[١٠٠٠٠] عَشَرَةُ آلَافٍ
            \item[١٥٠٠٠] خَمْسَةَ عَشَرَ أَلْفًا
            \item[٥٠٠٠٠] خَمْسُونَ أَلْفًا
            \item[١٠٠٠٠٠] مِائَةُ أَلْفٍ
            \item[٥٠٠٠٠٠] خَمْسُمِائَةِ أَلْفٍ
        \end{description}
    \end{multicols}
\end{Arabic}

\begin{note}
    带零碎就作区分语,几十也是区分语,虽然到现在也没解释什么是区分语。
\end{note}

\begin{attention}
    跟 \arm{٢٠٠} 规律一样, \arm{٢٠٠٠} 宾属格 \arm{أَلْفَيْنِ} ,后面接词的时候依然删 \arm{نِ} (宾属格 \arm{أَلْفَيْ}) 。
\end{attention}

\begin{Arabic}
    \begin{center}
        \begin{tabular}{c|cc}
            \crm{数字} & قَلَمٌ & مِقْلَمَةٌ \\
            \hline
            ١٠٠٠ & أَلْفُ \gray{قَلَمٍ}& أَلْفُ \gray{مِقْلَمَةٍ}\\
            ٢٠٠٠ & أَلْفَا \gray{قَلَمٍ} & أَلْفَا \gray{مِقْلَمَةٍ}\\
            ١٠٠٢ & أَلْفُ \gray{قَلَمٍ} وَقَلَمَانِ\\
            ١٥٠٠ & أَلْفٌ وَخَمْسُمِائَةِ \gray{قَلَمٍ}\\
            ٢٥٣٥ & أَلْفَانِ وَخَمْسُمِائَةٍ وَخَمْسَةٌ وَعِثْرُونَ \gray{قَلَمًا} \\
        \end{tabular}
    \end{center}
\end{Arabic}

\begin{itemize}
    \item \ac{أَلْفُ لَيْلَةٍ وَلَيْلَةٌ}{《一千零一夜》}
\end{itemize}

\subsection{$10^6$}

\begin{Arabic}
    \begin{multicols}{2}
        \begin{description}
            \item[١٠\textsuperscript{٦}] مِلْيُونٌ جـ مَلَايِينُ
            \item[٢\ae{٦}] مِلْيُونَانِ 
            \item[٣\ae{٦}] ثَلَاثَةُ مَلَايِينَ
            \item[٤\ae{٦}] أَرْبَعَةُ مَلَايِينَ
            \item[٥\ae{٦}] خَمْسَةُ مَلَايِينَ
            \item[٦\ae{٦}] سِتَّةُ مَلَايِينَ
            \item[٧\ae{٦}] سَبْعَةُ مَلَايِينَ
            \item[٨\ae{٦}] ثَمَانِيَةُ مَلَايِينَ
            \item[٩\ae{٦}] تِسْعَةُ مَلَايِينَ 
            \item[١\ae{٧}] عَشَرَةُ مَلَايِينَ
            \item[١٫٥\ae{٧}] خَمْسَةَ عَشَرَ مِلْيُونًا
            \item[٥\ae{٧}] خَمْسُونَ مِلْيُونًا
            \item[١\ae{٨}] مِائَةُ مِلْيُونٍ
            \item[٥\ae{٨}] خَمْسُمِائَةِ مِلْيُونٍ
        \end{description}
    \end{multicols}
\end{Arabic}

\begin{attention}
    \arm{مَلَايِينُ} 半变尾名词,宾属格 \arm{مَلَايِينَ} 。

    \arm{مِلْيُونَانِ} 宾属格 \arm{مِلْيُونَيْنِ} 。
\end{attention}

\begin{note}
    公式表示方法是上网搜的。其中,表示小数点的符号很特殊,是`` \arm{٫} ''(U+066B, Arabic Decimal Separator),不知道怎么在键盘上打出来,是复制粘贴的。千位分隔符另有码位`` \arm{٬} ''(U+066C, Arabic Thousands Separator),跟阿拉伯逗号码位不同。例如,\arm{١٢٣٬٤٥٦٬٧٨٩٫٠٠} 即$123,456,789.00$。
\end{note}

\subsection{$10^{9+}$}

\begin{Arabic}
    \begin{multicols}{2}
        \begin{description}
            \item[١٠\textsuperscript{٩}] مِلْيَارٌ جـ مِلْيَارَاتٌ
            \item[٢\ae{٩}] مِلْيَارَانِ
            \item[٣\ae{٩}] ثَلَاثَةُ مِلْيَارَاتٍ
            \item[٤\ae{٩}] أَرْبَعَةُ مِلْيَارَاتٍ
            \item[٥\ae{٩}] خَمْسَةُ مِلْيَارَاتٍ
            \item[٦\ae{٩}] سِتَّةُ مِلْيَارَاتٍ
            \item[٧\ae{٩}] سَبْعَةُ مِلْيَارَاتٍ
            \item[٨\ae{٩}] ثَمَانِيَةُ مِلْيَارَاتٍ
            \item[٩\ae{٩}] تِسْعَةُ مِلْيَارَاتٍ
            \item[١٠\textsuperscript{١٠}] عَشَرَةُ مِلْيَارَاتٍ
            \item[١٫٥\ae{١٠}] خَمْسَةَ عَشَرَ مِلْيَارًا
            \item[٥\ae{١٠}] خَمْسُونَ مِلْيَارًا
            \item[١٠\textsuperscript{١١}] مِائَةُ مِلْيَارٍ
            \item[١٠\textsuperscript{١٢}] تِرِيلِيُونٌ جـ تِرِيلِيُونَاتٌ
        \end{description}
    \end{multicols}
\end{Arabic}

\begin{attention}
    \arm{مِلْيَارٌ} 中的 \arm{ـيَـ} 发音浑变,似乎 \arm{يَار} 这样的组合都如此。

    \arm{مِلْيَارٌ} 的复数 \arm{مَلْيَارَاتٌ} ,课上说是标准的完整式阴性复数,但弹幕和评论区指出应该是破碎式的,并解释道:``因为十亿是阳性,三十亿的三也和十亿反阴阳。''暂没有进一步核实。

    \arm{مِلْيَارَانِ} 宾属格 \arm{مِلْيَارَيْنِ} 。
\end{attention}

\begin{note}
    课件上好像有些发音符号标错了。当然,这个笔记也不敢保证错得比课上少……
\end{note}
