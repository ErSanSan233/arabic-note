\chapter{数字}

\section{\lecon{14.1}认识数词}


阿拉伯印度数字的书写方向与阿拉伯数字完全相同。例如 2025 直接写作 \arm{٢٠٢٥}。

各数字名称如下:

\begin{Arabic}
    \begin{multicols}{2}
    \begin{description}
        \item[٠] صِفْرٌ
        \item[١] وَحِدٌ
        \item[٢] اِثْنَانِ
        \item[٣] ثَلاثَةٌ
        \item[٤] أَرْبَعَةٌ
        \item[٥] خَمْسَةٌ
        \item[٦] سِتَّةٌ
        \item[٧] سَبْعَةٌ
        \item[٨] ثَمَانِيَةٌ
        \item[٩] تِسْعَةٌ
        \item[١٠] عَشَرَةٌ
    \end{description}
    \end{multicols}
\end{Arabic}

〔法〕zéro/chiffre、〔英〕cipher等词与\arm{صِفْرٌ} ( \arm{٠} )同源。

\newfontfamily\handwritingarabicfont[
    Script=Arabic,
    ItalicFont=Mishafi,
    Scale=1.6 
]{Mishafi}

\begin{attention}
    \arm{٢} 可以写为 \handwritingarabicfont{٢}。
\end{attention}

\begin{note}
    此外,课上强调 \arm{١} 手写时需要倾斜,似乎倾斜是一项不能轻易舍弃的重要特征。 \arm{٢} 的另一种形态 \handwritingarabicfont{٢} 本质上是上端可以忽略向下的弧度,在艺术字中往往会呈现为横平竖直的折线,与 \arm{٦} 完全对称呈现。
\end{note}

\section{\lecon{14.2}\lecon{14.3}基数词1~10}

1、2位于被数名词后面,性、格、式需要配合。

\begin{Arabic}
    \begin{center}
        \begin{tabular}{c|ccc}
        \crm{数字} & \crm{主} & \crm{宾} & \crm{属} \\
        \hline
        ١ & وَاحِدٌ م وَحِدَةٌ& وَاحِدًا م وَحِدَةً & وَاحِدٍ م وَحِدَةٍ \\
        ٢ & اِثْنَانِ م اِشْنَتَانِ & \multicolumn{2}{c}{اِثْنَيْنِ م اِثْنَتَيْنِ} \\
    \end{tabular}
    \end{center}
\end{Arabic}

3~10位于被数名词前面,标单音符,性与被数名词\emph{相反}。此时,被数名词只能是复数属格泛指式,与基数词构成正偏组合。

以下是主格接 \arm{قَلَمٌ جـ أَقْلَامٌ} (阳性)及 \arm{مِقْلَمَةٌ جـ مَقَالِمُ} (阴性) 的例子:

\begin{attention}
    \arm{مَقَالِةُ} 词尾的 \arm{ـُ} \emph{表示}该词为半变尾名词,即其宾属格同形标 \arm{ـَ}。
\end{attention}

\begin{note}
    目前暂时不清楚词尾的 \arm{ـُ} 是否\emph{标志着}该词为半变尾名词。
\end{note}

\begin{Arabic}
    \begin{center}
        \begin{tabular}{c|c|ccl}
            \crm{数字} & \crm{本名} & قَلَمٌ & مِقْلَمَةٌ & \\
            \hline
            ١ & وَحِدٌ & \gray{قَلَمٌ} وَاحِدٌ & \gray{مِقْلَمَةٌ} وَاحِدَةٌ \\
            ٢ & اِثْنَانِ & \gray{قَلَمَانِ} اِثْنَانِ & \gray{مِقْلَمَتَانِ} اِثْنَتَانِ \\
            ٣ & ثَلاثَةٌ & ثَلَاثَةُ\red{$^*$} \gray{أَقْلَامٍ} & ثَلَاثُ \gray{مَقَالِمَ} \\
            ٤ & أَرْبَعَةٌ & أَرْبَعَةُ \gray{أَقْلَامٍ} & أَرْبَعُ \gray{مَقَالِمَ} \\
            ٥ & خَمْسَةٌ & خَمْسَةُ \gray{أَقْلَامٍ} & خَمْسُ \gray{مَقَالِمَ} \\
            ٦ & سِتَّةٌ & سِتَّةُ \gray{أَقْلَامٍ} & سِتُّ \gray{مَقَالِمَ} \\
            ٧ & سَبْعَةٌ & سَبْعَةُ \gray{أَقْلَامٍ} & سَبْعُ \gray{مَقَالِمَ} \\
            ٨ & ثَمَانِيَةٌ & ثَمَانِيَةٌ \gray{أَقْلَامٍ} & ثَمَانِي\red{$^\dagger$} \gray{مَقَالِمَ} \\
            ٩ & تِسْعَةٌ & تِسْعَةُ \gray{أَقْلَامٍ} & تِسْعُ \gray{مَقَالِمَ} \\
            ١٠ & عَشَرَةٌ & عَشَرَةُ \gray{أَقْلَامٍ} & عَشْرُ\red{$^\ddagger$} \gray{مَقَالِمَ} \\
        \end{tabular}
    \end{center}
\end{Arabic}

\begin{footnotesize}
\begin{itemize}
    \item [\red{$^*$}] 注意单音符,下同。
    \item [\red{$^\dagger$}] \arm{٨} 的阳性 \arm{شَمَانٍ} 为缺尾名词,词尾不表示格位;改单音符时,改为长音(即所谓``写出暗藏的 \arm{ى} ,把鼻音符打开'')。
    \item [\red{$^\ddagger$}] \arm{١٠} 的阳性 \arm{عَشْرٌ} 拼写特殊,注意不是 \arm{ـشَـ} 。
\end{itemize}
\end{footnotesize}

\begin{note}
    前面提了好几次``阳性指物名词复数当成阴性单数看待'',这里其实数字 3~10 恰好在某种程度上呼应了这样的规则。我觉得`` 3~9 反阴阳''实际是性的搭配``负负得正''的结果:本来数词就已经用阴性表示复数了,再配合被数名词的阴性,反而呈现成阳性的态势。

    此外,\arm{اِثْنَانِ} 本身也满足名词双数的变位规则。
\end{note}