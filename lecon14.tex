\chapter{数字}

\section{\lecon{14.1}认识数词}


阿拉伯印度数字的书写方向与阿拉伯数字完全相同。例如 2025 直接写作 \arm{٢٠٢٥}。

各数字名称如下:

\begin{Arabic}
    \begin{multicols}{2}
    \begin{description}
        \item[٠] صِفْرٌ
        \item[١] وَحِدٌ
        \item[٢] اِثْنَانِ
        \item[٣] ثَلاثَةٌ
        \item[٤] أَرْبَعَةٌ
        \item[٥] خَمْسَةٌ
        \item[٦] سِتَّةٌ
        \item[٧] سَبْعَةٌ
        \item[٨] ثَمَانِيَةٌ
        \item[٩] تِسْعَةٌ
        \item[١٠] عَشَرَةٌ
    \end{description}
    \end{multicols}
\end{Arabic}

〔法〕zéro/chiffre、〔英〕cipher等词与\arm{صِفْرٌ} ( \arm{٠} )同源。

\newfontfamily\handwritingarabicfont[
    Script=Arabic,
    ItalicFont=Mishafi,
    Scale=1.6 
]{Mishafi}

\begin{attention}
    \arm{٢} 可以写为 \handwritingarabicfont{٢}。
\end{attention}

\begin{note}
    此外,课上强调 \arm{١} 手写时需要倾斜,似乎倾斜是一项不能轻易舍弃的重要特征。 \arm{٢} 的另一种形态 \handwritingarabicfont{٢} 本质上是上端可以忽略向下的弧度,在艺术字中往往会呈现为横平竖直的折线,与 \arm{٦} 完全对称呈现。
\end{note}

\section{\lecon{14.2}\lecon{14.3}基数词1~10}

\begin{note}
    本课从此处开始,我觉得课上讲的规则有点过于复杂了。但考虑到这几节课的大多数资源都没有声音,这里先尽量多抄录,之后再整合。
\end{note}

1、2位于被数名词后面,性、格、式需要配合。

\begin{Arabic}
    \begin{center}
        \begin{tabular}{c|ccc}
        \crm{数字} & \crm{主} & \crm{宾} & \crm{属} \\
        \hline
        ١ & وَاحِدٌ م وَحِدَةٌ& وَاحِدًا م وَحِدَةً & وَاحِدٍ م وَحِدَةٍ \\
        ٢ & اِثْنَانِ م اِشْنَتَانِ & \multicolumn{2}{c}{اِثْنَيْنِ م اِثْنَتَيْنِ} \\
    \end{tabular}
    \end{center}
\end{Arabic}

3~10位于被数名词前面,标单音符,性与被数名词\emph{相反}。此时,被数名词只能是复数属格泛指式,与基数词构成正偏组合。

以下是主格接 \arm{قَلَمٌ جـ أَقْلَامٌ} (阳性)及 \arm{مِقْلَمَةٌ جـ مَقَالِمُ} (阴性) 的例子:

\begin{attention}
    \arm{مَقَالِةُ} 词尾的 \arm{ـُ} 表示该词为半变尾名词,即其宾属格同形标 \arm{ـَ}。
\end{attention}

\begin{note}
    目前暂时不清楚词尾的 \arm{ـُ} 是否标志着该词为半变尾名词。
\end{note}

\begin{Arabic}
    \begin{center}
        \begin{tabular}{c|c|ccl}
            \crm{数字} & \crm{本名} & قَلَمٌ & مِقْلَمَةٌ & \\
            \hline
            ١ & وَحِدٌ & \gray{قَلَمٌ} وَاحِدٌ & \gray{مِقْلَمَةٌ} وَاحِدَةٌ \\
            ٢ & اِثْنَانِ & \gray{قَلَمَانِ} اِثْنَانِ & \gray{مِقْلَمَتَانِ} اِثْنَتَانِ \\
            ٣ & ثَلاثَةٌ & ثَلَاثَةُ\red{$^*$} \gray{أَقْلَامٍ} & ثَلَاثُ \gray{مَقَالِمَ} \\
            ٤ & أَرْبَعَةٌ & أَرْبَعَةُ \gray{أَقْلَامٍ} & أَرْبَعُ \gray{مَقَالِمَ} \\
            ٥ & خَمْسَةٌ & خَمْسَةُ \gray{أَقْلَامٍ} & خَمْسُ \gray{مَقَالِمَ} \\
            ٦ & سِتَّةٌ & سِتَّةُ \gray{أَقْلَامٍ} & سِتُّ \gray{مَقَالِمَ} \\
            ٧ & سَبْعَةٌ & سَبْعَةُ \gray{أَقْلَامٍ} & سَبْعُ \gray{مَقَالِمَ} \\
            ٨ & ثَمَانِيَةٌ & ثَمَانِيَةٌ \gray{أَقْلَامٍ} & ثَمَانِي\red{$^\dagger$} \gray{مَقَالِمَ} \\
            ٩ & تِسْعَةٌ & تِسْعَةُ \gray{أَقْلَامٍ} & تِسْعُ \gray{مَقَالِمَ} \\
            ١٠ & عَشَرَةٌ & عَشَرَةُ \gray{أَقْلَامٍ} & عَشْرُ\red{$^\ddagger$} \gray{مَقَالِمَ} \\
        \end{tabular}
    \end{center}
\end{Arabic}

\begin{footnotesize}
\begin{itemize}
    \item [\red{$^*$}] 注意单音符,下同。
    \item [\red{$^\dagger$}] \arm{٨} 的阳性 \arm{شَمَانٍ} 为缺尾名词,词尾不表示格位;改单音符时,改为长音(即所谓``写出暗藏的 \arm{ى} ,把鼻音符打开'')。
    \item [\red{$^\ddagger$}] \arm{١٠} 的阳性 \arm{عَشْرٌ} 拼写特殊,注意不是 \arm{ـشَـ} 。
\end{itemize}
\end{footnotesize}

\begin{note}
    前面提了好几次``阳性指物名词复数当成阴性单数看待'',这里其实数字 3~10 恰好在某种程度上呼应了这样的规则。我觉得`` 3~9 反阴阳''实际是性的搭配``负负得正''的结果:本来数词就已经用阴性表示复数了,再配合被数名词的阴性,反而呈现成阳性的态势。

    此外,\arm{اِثْنَانِ} 本身也满足名词双数的变位规则。
\end{note}

\section{\lecon{14.4}基数词11~99}

位于被数名词之前,个位数 1~9 性依前。被数名词均为单数宾格泛指式,作基数词的\emph{区分语}。

\subsection{11~19}

个位在前,十位在后(拼写特殊),二者词尾都标 \arm{ـَ};个位性变化依前,十位性与被数名词一致;仅 12 有格的变化。形如:

\begin{center}
    \arm{\crm{个位}ــَ + \crm{十位}ــَ + \xcancel{الـ}ـ\crm{名词}ــًا}
\end{center}

\begin{Arabic}
    \begin{center}
        \begin{tabular}{c|cc}
            \crm{数字} & قَلَمٌ & مِقْلَمَةٌ \\
            \hline
            ١١ & أَحَدَ عَشَرَ \gray{قَلَمًا}& إِحْدَى عَشْرَةَ\red{$^*$} \gray{مِقْلَمَةً}\\
            ١٢\red{$^\dagger$} & اِثْنَا عَشَرَ \gray{قَلَمًا}& اِثْنَتَا عَشْرَةَ \gray{مِقْلَمَةً}\\
            ١٣ & ثَلَاثَةَ عَشَرَ\red{$^\ddagger$} \gray{قَلَمًا}& ثَلَاثَ عَشْرَةَ \gray{مِقْلَمَةً}\\
            ١٤ & أَرْبَعَةَ عَشَرَ \gray{قَلَمًا} & أَرْبَعَ عَشْرَةَ \gray{مِقْلَمَةً}\\
            ١٥ & خَمْسَةَ عَشَرَ \gray{قَلَمًا} & خَمْسَ عَشْرَةَ \gray{مِقْلَمَةً}\\
            ١٦ & سِتَّةَ عَشَرَ \gray{قَلَمًا} & سِتَّ عَشْرَةَ \gray{مِقْلَمَةً}\\
            ١٧ & سَبْعَةَ عَشَرَ \gray{قَلَمًا} & سَبْعَ عَشْرَةَ \gray{مِقْلَمَةً}\\
            ١٨ & ثَمَانِيَةَ عَشَرَ \gray{قَلَمًا} & ثَمَانِيَ عَشْرَةَ \gray{مِقْلَمَةً}\\
            ١٩ & تِسْعَةَ عَشَرَ \gray{قَلَمًا} & تِسْعَ عَشْرَةَ \gray{مِقْلَمَةً}\\
        \end{tabular}
    \end{center}
\end{Arabic}

\begin{footnotesize}
\begin{itemize}
    \item [\red{$^*$}] 注意 10 作十位数时阴阳性跟词中的 \arm{ـشـ} 发音符号搭配( \arm{ـشَـٌ م ـشْـةٌ} )跟单独用时 ( \arm{ـشْـٌ م ـشَـةٌ} ) 相反。
    \item [\red{$^\dagger$}] 有变格,展示的是主格,宾属格为 \arm{اِثْنَيْ عَشَرَ \gray{(قَلَمًا)} م اِثْنَتَيْ عَشْرَةَ \gray{(مِقْلَمَةً)}} 。同时,2 作 12 的个位数时,各格都去掉了词尾 \arm{نِ} 。
    \item [\red{$^\ddagger$}] \arm{٣} 为阴性,\arm{١} 为阳性,此时的 \arm{١٣} 整体上称作``阴性的 13''。
\end{itemize}
\end{footnotesize}

\subsection{10$\mathbb{Z} $}

位于被数名词前,不变性,只变格,同名词完整式阳性复数词尾变格相同。

\begin{Arabic}
    \begin{center}
        \begin{tabular}{c|cc}
            \crm{数字} & \crm{主格} & \crm{宾属格} \\
            \hline
            ٢٠ & عِثْرُونَ \gray{قَلَمًا/مِقْلَمَةً}& عِثْرِينَ \gray{قَلَمًا/مِقْلَمَةً}\\
            ٣٠ & ثَلاَثُونَ \gray{قَلَمًا/مِقْلَمَةً}& ثَلاَثِينَ \gray{قَلَمًا/مِقْلَمَةً}\\
            ٤٠ & أَرْبَعُونَ \gray{قَلَمًا/مِقْلَمَةً}& أَرْبَعِينَ \gray{قَلَمًا/مِقْلَمَةً}\\
            ٥٠ & خَمْسُونَ \gray{قَلَمًا/مِقْلَمَةً}& خَمْسِينَ \gray{قَلَمًا/مِقْلَمَةً}\\
            ٦٠ & سِتُّونَ \gray{قَلَمًا/مِقْلَمَةً}& سِتِّينَ \gray{قَلَمًا/مِقْلَمَةً}\\
            ٧٠ & سَبْعُونَ \gray{قَلَمًا/مِقْلَمَةً}& سَبْعِينَ \gray{قَلَمًا/مِقْلَمَةً}\\
            ٨٠ & ثَمَانُونَ \gray{قَلَمًا/مِقْلَمَةً}& ثَمَانِينَ \gray{قَلَمًا/مِقْلَمَةً}\\
            ٩٠ & تِسْعُونَ \gray{قَلَمًا/مِقْلَمَةً}& تِسْعِينَ \gray{قَلَمًا/مِقْلَمَةً}\\
        \end{tabular}
    \end{center}
\end{Arabic}

\subsection{其余}

\begin{center}
    \arm{\crm{个位} + وَ\crm{十位} + \xcancel{الـ}ـ\crm{名词}ــًا}
\end{center}

十位不变性,个位变性依前,均变格。

\begin{Arabic}
    \begin{center}
        \begin{tabular}{c|cc}
            \crm{数字} & قَلَمٌ & مِقْلَمَةٌ \\
            \hline
            ٢١ & وَاحِدٌ وَعِشْرُونَ \gray{قَلَمًا}&  وَاحِدَةٌ وَعِشْرُونَ \gray{مِقْلَمَةً}\\
            ٢٢ & اِثْنَانِ وَعِشْرُونَ \gray{قَلَمًا}& اِثْنَتَانِ وَعِشْرُونَ \gray{مِقْلَمَةً}\\
            ٣٣ & ثَلاثَةٌ وَثَلاَثُونَ \gray{قَلَمًا}& ثَلاثٌ وَثَلاَثُونَ \gray{مِقْلَمَةً}\\
            ٤٤ & أَرْبَعَةٌ وَأَرْبَعُونَ \gray{قَلَمًا}& أَرْبَعٌ وَأَرْبَعُونَ \gray{مِقْلَمَةً}\\
            ٥٥ & خَمْسَةٌ وَخَمْسُونَ \gray{قَلَمًا}& خَمْسٌ وَخَمْسُونَ \gray{مِقْلَمَةً}\\
            ٦٦ & سِتَّةٌ وَسِتُّونَ \gray{قَلَمًا}& سِتٌّ وَسِتُّونَ \gray{مِقْلَمَةً}\\
            ٧٧ & سَبْعَةٌ وَسَبْعُونَ \gray{قَلَمًا}& سَبْعٌ وَسَبْعُونَ \gray{مِقْلَمَةً}\\
            ٨٨ & ثَمَانِيَةٌ وَثَمَانُونَ \gray{قَلَمًا}& ثَمَانٍ وَثَمَانُونَ \gray{مِقْلَمَةً}\\
            ٩٩ & تِسْعَةٌ وَتِسْعُونَ \gray{قَلَمًا}& تِسْعٌ وَتِسْعُونَ \gray{مِقْلَمَةً}\\
        \end{tabular}
    \end{center}
\end{Arabic}

变格时,记得个位和十位同时变格:

\begin{Arabic}
    \begin{center}
        \begin{tabular}{c|cc}
            ٣١ & قَلَمٌ & مِقْلَمَةٌ \\
            \hline
            \crm{主} & وَاحِدٌ وَثَلاَثُونَ \gray{قَلَمًا} & وَاحِدَةٌ وَثَلاَثُونَ \gray{مِقْلَمَةً}\\
            \crm{宾} & وَاحِدًا وَثَلاَثِينَ \gray{قَلَمًا} & وَاحِدَةً وَثَلاَثِينَ \gray{مِقْلَمَةً}\\
            \crm{属} & وَاحِدٍ وَثَلاَثِينَ \gray{قَلَمًا} & وَاحِدَةٍ وَثَلاَثِينَ \gray{مِقْلَمَةً}\\
            \hline
            \hline
            ٣٥ \\
            \hline
            \crm{主} & خَمْسَةٌ وَثَلاَثُونَ \gray{قَلَمًا} & خَمْسٌ وَثَلاَثُونَ \gray{مِقْلَمَةً}\\
            \crm{宾} & خَمْسَةً وَثَلاَثِينَ \gray{قَلَمًا} & خَمْسًا وَثَلاَثِينَ \gray{مِقْلَمَةً}\\
            \crm{属} & خَمْسَةٍ وَثَلاَثِينَ \gray{قَلَمًا} & خَمْسٍ وَثَلاَثِينَ \gray{مِقْلَمَةً}\\
        \end{tabular}
    \end{center}
\end{Arabic}

\section{总结}

\begin{Arabic}
    \begin{center}
        \begin{tabular}{ccc}
            
        \end{tabular}
    \end{center}
\end{Arabic}