\chapter{数字}

\section{\lecon{14.1}认识数词}


阿拉伯印度数字的书写方向与阿拉伯数字完全相同。例如 2025 直接写作 \arm{٢٠٢٥}。

各数字名称如下:

\begin{Arabic}
    \begin{multicols}{2}
    \begin{description}
        \item[٠] صِفْرٌ
        \item[١] وَحِدٌ
        \item[٢] اِثْنَانِ
        \item[٣] ثَلاثَةٌ
        \item[٤] أَرْبَعَةٌ
        \item[٥] خَمْسَةٌ
        \item[٦] سِتَّةٌ
        \item[٧] سَبْعَةٌ
        \item[٨] ثَمَانِيَةٌ
        \item[٩] تِسْعَةٌ
        \item[١٠] عَشَرَةٌ
    \end{description}
    \end{multicols}
\end{Arabic}

〔法〕zéro/chiffre、〔英〕cipher等词与\arm{صِفْرٌ} ( \arm{٠} )同源。

\newfontfamily\handwritingarabicfont[
    Script=Arabic,
    ItalicFont=Mishafi,
    Scale=1.6 
]{Mishafi}

\begin{attention}
    \arm{٢} 可以写为 \handwritingarabicfont{٢}。
\end{attention}

\begin{note}
    此外,课上强调 \arm{١} 手写时需要倾斜,似乎倾斜是一项不能轻易舍弃的重要特征。 \arm{٢} 的另一种形态 \handwritingarabicfont{٢} 本质上是上端可以忽略向下的弧度,在艺术字中往往会呈现为横平竖直的折线,与 \arm{٦} 完全对称呈现。
\end{note}

