\chapter{\textmongolian{ᠪᡳ ᠣᠴᡳ ᡨᠠᠴᡳᠰᡳ ᡳᠨᡠ}}

\section{\lecon{10.1} 单词和课文}

\begin{itemize}
    \item \ac{هَلْ}{……吗?}
    \item \ac{طَالِبٌ / طَلِبَةٌ}{学生}
    \item \ac{طَبِيبٌ / طَبِيبَةٌ}{医生}
    \item \ac{صَدِيقٌ / صَدِيقَةٌ}{朋友}
    \item \ac{صَدِيقِى / صَدِيقَتِى}{我的朋友}
    \item \ac{زَمِيلٌ / زَمِيلَةٌ}{同学/同事}
    \item \ac{زَمِيلِى / زَمِيلَتِى}{我的同学/同事}
    \item \ac{يَا}{喂(后面人名末尾的 \arm{ـٌ} 变成 \arm{ـُ})}
    \item \ac{نَعَمْ}{是}
    \item \ac{لَا}{否}
    \item \ac{مُدَرِّسٌ / مُدِرِّسَةٌ}{老师}
    \item \ac{مُهَنْدِسٌ / مُهَنْدِسَةٌ}{工程师}
    \item \ac{مُمَرِّضٌ / مُمَرِّضَةٌ}{护士}
\end{itemize}

\begin{Arabic}
    - أَهْلاً وَسَهْلاً!

    - أَهْلاً بِكَ!

    - هَلْ أنْتَ طَالِبٌ؟

    - نَعَمْ. أَنَا طَالِبٌ. هَلْ أَنْتِ طَالِبَةٌ أَيْضًا؟

    - لَا، أَنَا طَبِيبَةٌ.

    - يَا يُمْنَى، هُوَ صَدِيقِى، اِسْمُهُ أَحّمَدُ.

    - أَهْلاً بِكَ، يَا أَحّمَدُ

    - أَهْلاً بِكِ!

    - هَلْ أَنْتَ طَالِبٌ أَيْضًا؟

    - نَعَمْ. أَنَا طَالِبٌ أَيْضًا. أَنَا زَهِيلُهُ.

    - نَعَمْ، هُوَ زَمِيلِى.
\end{Arabic}

\begin{attention}
    \arm{أَيْضا} 也,\arm{زَمِيلُهُ} 他的同学。
\end{attention}



\paragraph{\lecon{10.2} ……的……}

\begin{note}
    这是属格接尾人称代词的先导。
\end{note}

以 \arm{زَمِيلٌ / زَمِيلَةٌ} 为例。

\begin{center}
    \begin{Arabic}
    \begin{tabular}{r|c|rr}
         & \crm{变位方式} & 阳 & 阴 \\
        \hline
        \crm{原型} & & زَمِيلٌ & زَمِيلَةٌ \\
        أَنَا & ـٌ $\leftarrow$  ـِى  & زَمِيلِى & زَمِيلَتِى \\
        أَنْتَ &   &  &  \\
        أَنْتِ &  &  &  \\
        هُوَ & ـٌ $\leftarrow$ ـُهُ & زَمِيلُهُ & زَمِيلَتُهُ \\
        هِيَ & ـٌ $\leftarrow$ ـُهَا & زَمِيلُهَا & زَمِيلَتُهَا \\
    \end{tabular}
\end{Arabic}
\end{center}

\section{语法}

\subsection{\lecon{10.3} 名词双数}

主格:\arm{ـٌ} \tto \arm{ـَانِ}

宾格、属格:\arm{ـٌ} \tto \arm{ـَيْنِ}

\begin{center}
    \begin{tabular}{cc|cc}
        两个…… & & 主格 & 宾、属格 \\
        \hline
        房子 & \arm{بَيْتٌ} & \arm{بَيْتَانِ} & \arm{بَيْتَِيْنِ} \\
        男人 & \arm{رَجُلٌ} & \arm{رَجُلَانِ} & \arm{رَجُلَيْنِ} \\
        小伙子 & \arm{فَتًى} & \arm{فَتَيَانِ} & \arm{فَتَيَيْنِ} \\
        姑娘 & \arm{فَتَاةٌ} & \arm{فَتَاتَانِ} & \arm{فَتَاتَيْنِ} \\
        公司 & \arm{شَرِكَةٌ} & \arm{شَركَتَانِ} & \arm{شَركَتَيْنِ} \\
    \end{tabular}
\end{center}

所谓``弄假成真'',即诸如 \arm{فَتًى} 末尾 \arm{ى} 形的 \arm{أ} 直接作为 \arm{ى} 而处理成 \arm{ـيـ}。

\subsection{\lecon{10.4} 主格独立人称代词}

\begin{center}
    \begin{Arabic}
    \begin{tabular}{c|c|c|c}
        \crm{人称} & \crm{单数} & \crm{双数} & \crm{复数} \\
        \hline
        \crm{一} & أَنَا = أَنَا & \crm{无} & نَحْنُ = نَحْنُ \\
        \crm{二} & أَنْتَ / أَنْتِ & أَنْتُمَا = أَنْتُمَا & أَنْتُمْ / أَنْتُنَّ \\
        \crm{三} & هُوَ / هِيَ & هُمَا = هُمَا & هُمْ / هُنَّ 
    \end{tabular}
\end{Arabic}
\end{center}

\subsection{\lecon{10.5} 属格接尾人称代词}

\begin{itemize}
    \item \ac{مِنْ}{prep. 来自}
    \item \ac{عَلَى}{在……上}
\end{itemize}

不独立使用,后缀与名词词尾。后缀变位:

\begin{center}
    \begin{Arabic}
        \begin{tabular}{c|c|c|c}
            \crm{人称} & \crm{单数} & \crm{双数} & \crm{复数} \\
            \hline
            \crm{一} & أَنَا : ـِى  & \crm{无} & نَحْنُ : ـنَا \\
            \hline
            \multirow{2}{*}{\crm{二}} & أَنْتَ : ـكَ & \multirow{2}{*}{أَنْتُمَا : ـكُمَا} & أَنْتُمْ : ـكُمْ \\
                & أَنْتِ : ـكِ & & أَنْتُنَّ : ـكُنَّ \\

            \hline
            \multirow{2}{*}{\crm{三}} & هُوَ : ـهُ & \multirow{2}{*}{هُمَا : ـهُمَا} & هُمْ : ـهُمْ \\
                & هِىَ : ـهَا & & هُنَّ : ـهُنَّ \\
        \end{tabular}
    \end{Arabic}
\end{center}

\begin{note}
    不知道为什么,课上喜欢从更为复杂的第三人称开始,讲到较为简单的第一人称。但笔记仍然遵循一、二、三人称的顺序了。

    具体来说,第三人称比较好记,双、复数完全就是直接把独立主格人称代词放到后面,单数却是一个短音一个长音。

    如果说第三人称的主题是 \arm{ه},那第二人称的主题就是 \arm{ك}。用满语来辅助记忆:双、复数直接就是``去掉第三人称的圈''。

    单数和第一人称硬记。

    属格不改变原词的性。如 \arm{بَيْتٌ}(阳性)接阴性后缀后(如 \arm{بَيْتُكِ})仍为阳性。
\end{note}

\paragraph{鼻音符不能接属格接尾人称代词} 去掉 \arm{ـً، ـٍ، ـٌ} 的双写,或直接被第一人称单数的 \arm{ـِى} 替代。

\paragraph{冠词和属格接尾人称代词不能同时存在} 很好理解,ma maison可以但la ma maison不可以。

\paragraph{第三人称音变} 以下结尾的名词变第三人称属格时 \arm{ـهُ/ـهُـ} 变 \arm{ـهِ/ـهِـ}:

\begin{Arabic}
    \begin{itemize}
        \item ـِ
        \item ـِى
        \item ـَىْ
    \end{itemize}
\end{Arabic}

以 \arm{بَيْتٌ} 的词组 \arm{فِي بَيْتِ}(在房子里)为例:

\begin{center}
    \begin{Arabic}
        \begin{tabular}{c|c|c|c}
            \crm{在房子里} & \crm{第三人称单数} & \crm{第三人称双数} & \crm{第三人称复数} \\
            \hline
            \multirow{2}{*}{فِي بَيْتِ} & هُوَ : فِي بَيْتِهِ & \multirow{2}{*}{هُمَا : فِي بَيْتِهِمَا} & هُمْ : فِي بَيْتِهِمْ \\
                & هِىَ : فِي بَيْتِهَا & & هُنَّ : فِي بَيْتِهِنَّ \\
        \end{tabular}
    \end{Arabic}
\end{center}

\begin{note}
    其实就是 ``真[i]'' (而不含 \arm{ى} 形 \arm{ا} 这样的``假[i]'')结尾,真的很像元音和谐律。
\end{note}

\paragraph{附加的 \arm{ن} 接属格接尾人称代词要去掉} 例如名词双数、阳性完整式复数的情况。

\begin{note}
    记得连着 \arm{ن} 的元音一起去掉,即去掉整个音节。
\end{note}

\begin{note}
    \begin{center}
        % \begin{center}

\begin{tikzpicture}[
    scale=3,              % 图形缩放比例
    every node/.style={}, % 所有节点字体大小
    vertex/.style={circle, fill=black, inner sep=1.5pt}, % 顶点样式
    edge label/.style={midway, fill=white, inner sep=1pt}, % 边注释样式
]

% 定义正方体顶点坐标(三维坐标:x, y, z)
\coordinate (A) at (0,0,0);
\coordinate (B) at (-1,0,0);
\coordinate (C) at (-1,-1,0);
\coordinate (D) at (0,-1,0);
\coordinate (E) at (0,0,-1);
\coordinate (F) at (-1,0,-1);
\coordinate (G) at (-1,-1,-1);
\coordinate (H) at (0,-1,-1);

\draw (A) node[vertex, label=below right:\arm{بَيْتٌ} {\footnotesize 房}] {};
\draw (B) node[vertex, label=left:{\footnotesize 俩房} \arm{بَيْتَانِ}] {};
\draw (D) node[vertex, label=right:\arm{بَيْتُهُ} {\footnotesize 他的房}] {};
\draw (C) node[vertex, label=left:{\footnotesize 他的俩房} \arm{بَيْتَاهُ}] {};

\draw (E) node[vertex, label=right:\arm{فِى بَيْتِِ} {\footnotesize 房里}] {};
\draw (F) node[vertex, label=left:{\footnotesize 俩房里} \arm{فِى بَيْتَيْنِ}] {};
\draw (H) node[vertex, label=right:\arm{فِى بَيْتِهِ} {\footnotesize 他房里}] {};
\draw (G) node[vertex, label=below:{\footnotesize 他的俩房里} \arm{فِى بَيْتَيْهِ}] {};

\draw[] (C) -- (D);
\draw[dashed] (E) -- (F);
\draw[dashed] (G) -- (H);
\draw[dashed] (H) -- (E);

\draw[-{Stealth[length=3mm]}] (B) -- (C) node[edge label, sloped, right, rotate=180]{\footnotesize (去 \arm{ن})};
\draw[-{Stealth[length=3mm]}, dashed] (F) -- (G)node[edge label, sloped, left, rotate=180]{\footnotesize (去 \arm{ن})};

\draw[-{Stealth[length=3mm]}] (A) -- (B) node[edge label]{双};
\draw[-{Stealth[length=3mm]}] (A) -- (D) node[edge label, sloped, rotate=180]{属他};
\draw[-{Stealth[length=3mm]}, dashed] (A) -- (E) node[edge label, sloped]{宾};

\end{tikzpicture}

% \end{center}
    \end{center}
\end{note}

\begin{note}
    \arm{بَيْتٌ} \tto 宾格泛指 \arm{فِى بَيْتٍ},加属格接尾人称代词``在他的房子里''步骤如下:
    \begin{enumerate}
        \item 结尾不能鼻音符:\arm{فِى بَيْتٍ} \tto \arm{فِى بَيْتِ};
        \item \arm{ـِ} 结尾,\arm{ـهُ} 变 \arm{ـهِ}:\arm{فِى بَيْتِهُ} \tto \arm{فِى بَيْتِهِ}。
    \end{enumerate}
\end{note}

(还没学到)再如,完整式阳性复数加接尾人称代词:

\begin{itemize}
    \item 主格: \arm{ـُونَ} \tto \arm{ـُوـ..}
    \item 宾属格: \arm{ـِينَ}  \tto \arm{ـِيـ..}
\end{itemize}

\begin{Arabic}
    \begin{center}
        \begin{tabular}{c|cc}
            \crm{老师} & \crm{老师们} & \crm{她的老师们} \\
            \hline
            مُدَرِّسٌ & مُدَرِّسُونَ & مُدَرِّسُوهَا \\
             & مِنْ مُدَرِّسِينَ & مِنْ مُدَرِّسِيهَا
        \end{tabular}
    \end{center}
\end{Arabic}



\paragraph{第一人称读法变化} ``我的''接在下列词尾后时,不吞掉原词尾,直接加 \arm{ـىَ}

\begin{Arabic}
    \begin{itemize}
        \item ـَا
        \item ـُو
        \item ـَى
    \end{itemize}
\end{Arabic}

\begin{itemize}
    \item \ac{بَيْتَانِ \tto بَيْتَاىَ \tto عَلَى بَيْتَيْىَ}{俩房 \tto 我的俩房 \tto 我的俩房上}
    \begin{itemize}
        \item 第一个变化是先去 \arm{ن} 再由于 \arm{ـتَا} 而接 \arm{ـىَ}
        \item 第二个变化是先去 \arm{ن} 再由于 \arm{ـتَىْ} 而接 \arm{ـىَ}
    \end{itemize}
    \item \ac{مُدَرِّسُونَ \tto مُدَرِّسُوىَ \tto مِنْ مُدَرِّسِيىَ}{老师们 \tto 我老师们  \tto 来自我的老师们 \\}
\end{itemize}


在以下介词后时,也有读音变化:

\begin{itemize}
    \item \ac{بِ \tto بِيَ}{凭借、在 \tto 凭借我、在我这}
    \item \ac{فِي \tto فِيَّ}{在……里 \tto 在我的里面}
    \item \ac{مِنْ \tto مِنِّي}{从 \tto 从我这}
    \item \ac{عَنْ \tto عَنِّي}{关于、通过 \tto 关于我、通过我}
\end{itemize}

\begin{note}
    注意,指的是在两个介词后面直接变位(如``凭借我'')。此外,\arm{فِي} 和变音后缀的 \arm{ـنِي} 的长音下面是加两点的,不知道为什么。

    此外,我是真没想到介词也能这么变。
\end{note}