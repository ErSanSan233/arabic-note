\chapter{嘟、嘟、嘟、嘟、嘟、滴——}

\section{\lecon{15.1} 序数词}

阿语序数词和被数名词的关系是定语和被修饰语的关系。

序数词放在被数名词后面。通常用来形容确指名词。

\subsection{第1~第10}

性(不反阴阳)、格配合。

\begin{Arabic}
    \begin{center}
        \begin{tabular}{c|ccc}
        \crm{第……} & \crm{主} & \crm{宾} & \crm{属} \\
        \hline
        ال١\red{$^*$} & اَلْأَوَّلُ م اَلْأُولَى & اَلْأَوَّلَ م اَلْأُولَى & اَلْأَوَّلِ م اَلْأُولَى \\
        ال٢\red{$^\dagger$} & اَلثَّانِي م اَلثَّانِيةُ & اَلثَّانِيَ م اَلثَّانِيةَ & اَلثَّانِي م اَلثَّانِيةِ 
    \end{tabular}
    \end{center}
\end{Arabic}

\begin{footnotesize}
\begin{itemize}
    \item [\red{$^*$}] 第1的阴性 \arm{اَلْأُولَى} 是定尾名词。
    \item [\red{$^\dagger$}] 第2的阳性主、属格相同。
\end{itemize}
\end{footnotesize}

    以下序数词是规律变化:

\begin{Arabic}
    \begin{multicols}{3}
        \begin{description}
            \item[ال٣]  اَلثَّالِثُ م اَلثَّالِثَةُ
            \item[ال٤]  اَلرَّابِعُ م الرَّابِعَةُ
            \item[ال٥]  اَلْخَامِسُ م اَلْخَامِسَةُ
            \item[ال٦]  اَلسَّادِسُ م اَلسَّادِسَةُ
            \item[ال٧]  اَلسَّابِعُ م السَّابِعَةُ
            \item[ال٨]  اَلثَّامِنُ م اَلثَّامِنَةُ
            \item[ال٩]  اَلتَّاسِعُ م اَلتَّاسِعَةُ
            \item[ال١٠] اَلْعَاشِرُ م اَلْعَاشِرَةُ
            \item[~] 
        \end{description}
    \end{multicols}
\end{Arabic}

\subsection{第11~第19}

序数词第1~第9+基数词10。性一致,词尾均标开口符。不变格。

\begin{Arabic}
    \begin{multicols}{2}
        \begin{description}
            \item[ال١١] اَلْحَادِيَ عَشَرَ م اَلْحَادِيَةُ عَشْرَةَ
            \item[ال١٢] اَلثَّانِيَ عَشَرَ م اَلثَّانِيَةَ عَشْرَةَ
            \item[ال١٣] اَلثَّالِثَ عَشَرَ م اَلثَّالِثَةَ عَشْرَةَ
            \item[ال١٤] اَلرَّابِعَ عَشَرَ م الرَّابِعَةَ عَشْرَةَ
            \item[ال١٥] اَلْخَامِسَ عَشَرَ م اَلْخَامِسَةَ عَشْرَةَ
            \item[ال١٦] اَلسَّادِسَ عَشَرَ م اَلسَّادِسَةَ عَشْرَةَ
            \item[ال١٧] اَلسَّابِعَ عَشَرَ م السَّابِعَةَ عَشْرَةَ
            \item[ال١٨] اَلثَّامِنَ عَشَرَ م اَلثَّامِنَةَ عَشْرَةَ
            \item[ال١٩] اَلتَّاسِعَ عَشَرَ م اَلتَّاسِعَةَ عَشْرَةَ
            \item[~] 
        \end{description}
    \end{multicols}
\end{Arabic}

\subsection{第$10\mathbb{N} $}

基数词20~90直接加冠词,不变性,只变格。

\begin{Arabic}
    \begin{center}
        \begin{tabular}{c|cc}
            \crm{数字} & \crm{主格} & \crm{宾、属格} \\
            \hline
            ال٢٠ & اَلْعِشْرُونَ & اَلْعِشْرِينَ \\
            ال٣٠ & اَلثَّلاَثُونَ & اَلثَّلاَثِينَ \\
            ال٤٠ & اَلْأَرْبَعُونَ & اَلْأَرْبَعِينَ \\
            ال٥٠ & اَلْخَمْسُونَ & اَلْخَمْسِينَ \\
            ال٦٠ & اَلسِّتُّونَ & اَلسِّتِّينَ \\
            ال٧٠ & اَلسَّبْعُونَ & اَلسَّبْعِينَ \\
            ال٨٠ & اَلثَّمَانُونَ & اَلثَّمَانِينَ \\
            ال٩٠ & اَلتِّسْعُونَ & اَلتِّسْعِينَ \\
        \end{tabular}
    \end{center}
\end{Arabic}

\subsection{第$10\mathbb{N} + \mathbb{N}$}

\begin{center}
    \arm{الـ\crm{名词} + الـ\crm{个位变性变格} + وَالـ\crm{十位只变格}}
\end{center}

\begin{Arabic}
    \begin{description}
        \item[ال٢١] اَلْحَادِي وَالْعِشْرُونَ م اَلْحَادِيَةُ وَالْعِشْرُونَ  
        \item[ال٢٢] اَلثَّانِي وَالْعِشْرُونَ م اَلثَّانِيةُ وَالْعِشْرُونَ  
        \item[ال٣٣] اَلثَّالِثُ وَالثَّلاَثُونَ م اَلثَّالِثَةُ وَالثَّلاَثُونَ
        \item[ال٤٤] اَلرَّابِعُ وَالْأَرْبَعُونَ م الرَّابِعَةُ وَالْأَرْبَعُونَ
        \item[ال٥٥] اَلْخَامِسُ وَالْخَمْسُونَ م اَلْخَامِسَةُ وَالْخَمْسُونَ 
        \item[ال٦٦] اَلسَّادِسُ وَالسِّتُّونَ م اَلسَّادِسَةُ وَالسِّتُّونَ   
        \item[ال٧٧] اَلسَّابِعُ وَالسَّبْعُونَ م السَّابِعَةُ وَالسَّبْعُونَ  
        \item[ال٨٨] اَلثَّامِنُ وَالثَّمَانُونَ م اَلثَّامِنَةُ وَالثَّمَانُونَ 
        \item[ال٩٩] اَلتَّاسِعُ وَالتِّسْعُونَ م اَلتَّاسِعَةُ وَالتِّسْعُونَ  
    \end{description}
\end{Arabic}

% \begin{Arabic}
%     \begin{center}
%         \begin{tabular}{c|lcr}
%             \crm{数字} & \multicolumn{3}{c}{\crm{主格}} \\
%             \hline
%             ال٢١ & اَلْحَادِي وَالْعِشْرُونَ &م& اَلْحَادِيَةُ وَالْعِشْرُونَ  \\
%             ال٢٢ &  اَلثَّانِي وَالْعِشْرُونَ &م& اَلثَّانِيةُ وَالْعِشْرُونَ  \\
%             ال٣٣ & اَلثَّالِثُ وَالثَّلاَثُونَ &م& اَلثَّالِثَةُ وَالثَّلاَثُونَ \\
%             ال٤٤ & اَلرَّابِعُ وَالْأَرْبَعُونَ &م& الرَّابِعَةُ وَالْأَرْبَعُونَ \\
%             ال٥٥ & اَلْخَامِسُ وَالْخَمْسُونَ &م& اَلْخَامِسَةُ وَالْخَمْسُونَ  \\
%             ال٦٦ & اَلسَّادِسُ وَالسِّتُّونَ &م& اَلسَّدِسَةُ وَالسِّتُّونَ   \\
%             ال٧٧ & اَلسَّابِعُ وَالسَّبْعُونَ &م& السَّابِعَةُ وَالسَّبْعُونَ  \\
%             ال٨٨ & اَلثَّامِنُ وَالثَّمَانُونَ &م& اَلثَّامِنَةُ وَالثَّمَانُونَ \\
%             ال٩٩ & اَلتَّاسِعُ وَالتِّسْعُونَ &م& اَلتَّاسِعَةُ وَالتِّسْعُونَ  \\
%         \end{tabular}
%     \end{center}
% \end{Arabic}

\subsection{百、千及更高}

基数词加冠词即可。只变格,不变性。

\begin{Arabic}
    \begin{multicols}{2}
        \begin{description}
            \item[ال١٠٠] اَلْمِائَةُ 
            \item[ال١٠٠٠] اَلْأَلْفُ
        \end{description}
    \end{multicols}
\end{Arabic}

\subsection{举例}

\begin{itemize}
    \item \ac{دَرْسٌ}{课程}
    \item \ac{سَاعَةٌ}{小时,钟表,(加序数词)点钟}
\end{itemize}

\begin{Arabic}
    \begin{center}
        \begin{tabular}{c|cc}
            \crm{第……} & دَرْسٌ & سَاعَةٌ \\
            \hline
            ال٥ & \gray{اَلدَّرْسُ} الْخَامِسُ & \gray{اَلسَّاعَةُ} الْخَامِسَةُ \\
            ال١٢ & \gray{اَلدَّرْسُ} الثَّانِيَ عَشَرَ & \gray{اَلسَّاعَةُ} الثَّانِيَةَ عَشْرَةَ \\
            ال٢٠ & \gray{اَلدَّرْسُ} الْعِشْرُونَ\red{$^*$} & \gray{اَلسَّاعَةُ} الْعِشْرُونَ\\ 
            ال٢٤ & \gray{اَلدَّرْسُ} الرَّابِعُ وَالْعِشْرُونَ\red{$^\dagger$}& \gray{اَلسَّاعَةُ} الرَّابِعَةُ وَالْعِشْرُونَ\\
        \end{tabular}
    \end{center}
\end{Arabic}

\begin{footnotesize}
\begin{itemize}
    \item [\red{$^*$}] 宾格 \arm{اَلدَّرْسَ اَلْعِشْرِينَ} ;属格 \arm{اَلدَّرْسِ اَلْعِشْرِينَ} 。
    \item [\red{$^\dagger$}]  宾格 \arm{اَلدَّرْسَ الرَّابِعَ وَالْعِشْرِينَ} ;属格 \arm{اَلدَّرْسِ الرَّابِعِ وَالْعِشْرِينَ} 。
\end{itemize}
\end{footnotesize}

\begin{itemize}
    \item \ac{اَلْكِتَابُ الْمِائَةُ}{第100本书}
    \item \ac{اَللَّيْلَةُ الْأَلْفُ}{第1000个夜晚}
\end{itemize}