\chapter{卧槽,学霸!}

\section{\lecon{12.2} 单词和课文}

\begin{itemize}
    \item \ac{كَيْفَ}{n. 怎么样}
    \item \ac{حَالٌ}{情况}
    \item \ac{خَيْرٌ}{最好的}
    \item \ac{بِخَيْرٍ}{……很好(\arm{بِـ} 后属格,因此是 \arm{ـٍ})}
    \item \ac{شُكْرًا}{谢谢}
    \item \ac{قَلَمٌ جـ أَقْلَامٌ}{笔}
    \item \ac{مِقْلَمَةٌ جـ مَقَالِمُ}{笔袋}
    \item \ac{كِتَابٌ جـ كُتُبٌ}{书}
    \item \ac{دِرَاسَةٌ}{n. 学习(表抽象意义的词根形式)}
    \item \ac{لِمَنْ ...؟}{……是谁的?}
    \item \ac{... لِي.}{……是我的。}
    \item \ac{يَالَكَ مِنْ ...!}{你真是个……的人啊!}
    \item \ac{مُجْتَهِدٌ فِي ...}{在……上努力的人}
    \item \ac{بَابٌ جـ أَبْوَابٌ}{m. 门}
    \item \ac{نَافِذَةٌ جـ نَوَافِذُ}{f. 窗}
    \item \ac{مَكْتَبٌ جـ مَكَايِبُ}{m. 课桌}
    \item \ac{كُرْسِيٌّ جـ كَرَاسٍ }{(缺尾)m. 椅}
    \item \ac{اِسْمٌ جـ أَسْمَءٌ}{名字}
\end{itemize}

\begin{attention}
    \arm{جـ} 表示后侧是前侧的破碎式复数。

    书等指物名词复数语法上当成阴性单数看待。

    \begin{Arabic}
        \begin{itemize}
            \item هَزِهِ أَبْوَابٌ.
        \end{itemize}
    \end{Arabic}

    复数的 \arm{كَرَاسٍ} 为所谓``缺尾名词'',即其词尾永远是 \arm{ـٍ},并不代表它是属格。
\end{attention}

\begin{note}
    似乎只有破碎式复数而没有``破碎式双数''?
    \begin{itemize}
        \item \ac{هَزَانِ بَابَانِ.}{这是两扇门。}
    \end{itemize}
\end{note}

\begin{Arabic}
    - أَهْلًا وَسَهْلًا!

    - أَهْلًا بِكَ.

    - كَيْفَ حَالُكَ؟

    - أَنَا بِخَيْرٍ، شُكْرًا. وَأَنْتَ؟

    - أَنَا بِخَيْرٍ.

    - مَا هَزَا؟

    - هَزَا قَلَمٌ.

    - وَ مَا هَزِهِ؟

    - هَزِهِ مِقْلَمَةٌ.

    - وَ مَا هَزَا؟

    - هَزَا كِتَابٌ.

    - وَمَا هَزِهِ؟

    - هَزِهِ كُتُبٌ.

    - لِمَنْ هَزِهِ الْكُتُبُ؟

    - هَزِهِ الْكُتُبُ لِي.

    - يَا لَكِ مِنْ مُجْتَهِدَةٍ فِي الدِّرَاسَةِ!
\end{Arabic}

\paragraph{某物是谁的?} \arm{لِـ}--疑问代词(此课只有 \arm{لِمَنْ})+指示代词+ \arm{الـ}--物(注意词尾 \arm{ـٌ} 改成 \arm{ـُ})。

\paragraph{某物是……的} 指示代词+ \arm{الـ}--物+ \arm{لِـ}--属格接尾人称代词。

\begin{Arabic}
    \begin{itemize}
        \item هَزَا الْقَلَمُ / هَزِهِ الْمِقْلَمَةُ ...
        \begin{itemize}[label=\crm{--}]
            \item لِي
            \item لَكَ
            \item لَكِ
            \item لَهُ
            \item لَهَا
        \end{itemize}
    \end{itemize}
\end{Arabic}

\paragraph{你真是……的人啊!}  两种说法:

\begin{itemize}
    \item 感叹虚词+ \arm{لِـ}--人称代词 + \arm{مِنْ} + 名词\emph{属格泛指}。
    \item 感叹虚词+ \arm{لِـ}--人称代词 + 名词\emph{宾格泛指}。
\end{itemize}

\begin{Arabic}
    \begin{itemize}
        \item يَا لَكَ مِنْ رَجُلٍ! = يَا لَكَ رَجُلًا!
        \item يَا لَكَ مِنْ مُجْتَهِدٍ فِي الدِّرَاسَةِ! = يَا لَكَ مُجْتَهِدًا فِي الدِّرَاسَةِ!
    \end{itemize}
\end{Arabic}

\section{语法}

\subsection{\lecon{12.3} 泛指和确指}

\begin{note}
    没什么好记的,同直觉一致。
\end{note}

\subsection{\lecon{12.4} 冠词的发音}

\begin{Arabic}
    \begin{center}
        \begin{tabular}{c|cc}
            & ا & ل \\
            \hline
            \crm{发音} & \crm{前面没有东西}  & \crm{后面是太阴字母} \\
            \crm{不发音} & \crm{前面有东西} & \crm{后面是太阳字母}\\
        \end{tabular}
    \end{center} 
\end{Arabic}

\arm{ل} 不发音时,后面的太阳字母读叠音。

太阴字母:\arm{أ ب ج ح خ ع غ ف ق ك م ه و ى}

太阳字母:\arm{ت ث د ذ ر ز س ش ص ض ط ث ل ن}

\begin{note}
    太阴/太阳字母和前面冠词的 \arm{ل} 是否发音似乎是最讨厌的循环定义。不过,可以从字形上来简单记忆:

    \begin{center}
    \begin{tabular}{c|cc}
        & 太阴/ \arm{لْ} 发音 & 太阳/叠音+ \arm{ل} 不发音 \\
        \hline
        不规则 & 全部 & 无 \\
        短牙型 & 其他(带 \arm{ء}、下面带点) & 上面加点 \\
        长牙型 & 带 \arm{ء} 的 & 其他(\arm{ل}) \\
        闪电型 & 全部 & 无 \\
        小钩型 & 无 & 全部 \\
        横线型 & 无 & 全部 \\
        上圈型 & 只有圈点的 & 圈上带杠、牙的 \\
        三角型 & 全部 & 无
    \end{tabular}
    \end{center}

    即:

    \begin{center}
    \begin{tabular}{c|cc}
        & 太阴/ \arm{لْ} 发音 & 太阳/叠音+ \arm{ل} 不发音 \\
        \hline
        不规则 & \arm{ه و م} & \arm{} \\
        短牙型 & \arm{ـئـبـيـ} & \arm{ـنـتـثـ} \\
        长牙型 & \arm{ك} & \arm{ل} \\
        闪电型 & \arm{ـجـحـخـ} & \arm{} \\
        小钩型 & \arm{} & \arm{دذرز} \\
        横线型 & \arm{} & \arm{ـسـشـ} \\
        上圈型 & \arm{ـفـقـ} & \arm{ـصـضـطـظـ} \\
        三角型 & \arm{ـغـعـ} & \arm{} 
    \end{tabular}
    \end{center}
\end{note}

\subsection{\lecon{12.5} 名词的破碎式复数}

一些变尾名词的破碎式复数是半变尾名词。

\begin{itemize}
    \item \ac{زَمِيلٌ جـ زُمَلَاءُ}{同学}
    \item \ac{طَبِيبٌ جـ أَطِبَّاءُ}{医生}
    \item \ac{صَدِيقٌ جـ أَصْدِقَاءُ}{朋友}
\end{itemize}

\begin{note}
    课中有说半变尾名词意味着 \arm{زُمَلَاءُ}、\arm{أَطِبَّاءُ}、\arm{أَصْدِقَاءُ} 的宾、属格分别为 \arm{زُمَلَاءَ}、 \arm{أَطِبَّاءَ}、 \arm{أَصْدِقَاءَ}。没说为啥。
\end{note}

一些指人的阳性名词,破碎式复数带有 \arm{ة},但仍为阳性。

\begin{itemize}
    \item \ac{أُسْتَاذٌ جـ أَسَاتِذَةٌ}{教授}
    \item \ac{طَالِبٌ جـ طَلَبَةٌ}{学生}
\end{itemize}

一些名词有多种破碎式复数。

\begin{Arabic}
    \begin{itemize}
        \item طَالِبٌ جـ طَلَبَةٌ \& طُلَّابٌ
    \end{itemize}
\end{Arabic}

一些外来词的复数也是破碎的。

\begin{itemize}
    \item \ac{بَنْكٌ جـ بُنُوكٌ}{银行}
    \item \ac{فِلْمٌ جـ أَفْلَامٌ}{电影}
    \item \ac{قَيْصَرٌ جـ قَيَاصِرَةٌ}{凯撒/沙皇}
\end{itemize}

\begin{note}
    对于总结规律记忆破碎式复数,课上举了两组例子,抄录如下。
    \begin{Arabic}
        \begin{itemize}
            \item ـ ـ ـ جـ أَـْ ـَاـٌ
            \begin{itemize}[label=\crm{--}]
                \item فِلْمٌ جـ أَفْلَامٌ
                \item قَلَمٌ جـ أَقْلَامٌ
                \item اِسْمٌ جـ أَسْمَءٌ
            \end{itemize}
            \item ـ ـْ ـَ ـٌ جـ ـَ ـَاـِ ـَةٌ
            \begin{itemize}[label=\crm{--}]
                \item أُسْتَاذٌ جـ أَسَاتِذَةٌ
                \item قَيْصَرٌ جـ قَيَاصِرَةٌ
            \end{itemize}
        \end{itemize}
    \end{Arabic}
\end{note}

\subsection{\lecon{12.6} 名词句}

词首是名词就是名词句,判断句首时忽略某些虚词。

名词句的两个最基本的要素:

\begin{description}
    \item[起语] 就是接は的那东西。主格,确指;倒装名词句中为泛指。
    \item[述语] 就是接です的东西。主格,尽量泛指;意义上不能泛指也可以确指。
\end{description}

上述的主格指的是处于主格的地位,因为有时无法从变位看出具体的格。记得适当的时候性数配合。

\begin{note}
    俩都是主格。
\end{note}

\begin{itemize}
    \item \ac{أَنَا بِخَيْرٍ.}{我很好。}
    \item \ac{هَذِهِ الْكُتُبُ لِي.}{这些书是我的。}
    \item \ac{هِيَ جَمِيلَةٌ.}{她美丽。}
\end{itemize}

介词短语 \arm{بِخَيْرٍ}、 \arm{لِي} 看不出格位,但处于主格地位。半主动名词 \arm{جَمِيلَةٌ} 当形容词来用。