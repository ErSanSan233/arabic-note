\chapter{君の名は?}

\lecon{9.1}

\section{单词和课文}

\begin{itemize}
    \item \ac{مَا}{谁}
    \item \ac{اِسْمٌ}{名字}
    \item \ac{اِسْمُكَ / اِسْمُكِ}{你的名字}
    \item \ac{اِسْمِى}{我的名字}
    \item \ac{اِسْمُهُ / اِسْمُهَا}{他的名字}
    \item \ac{بِ}{prep. 借助}
    \item \ac{أَهْلََا بِكَ!}{你好}
    \item \ac{حَسَنََا}{好的}
    \item \ac{إِلَى}{prep. 向、到}
    \item \ac{اَلْ}{冠词(无\arm{ء
    })}
    \item \ac{لِقَاءٌ}{n. 见面}
    \item \ac{إِلَى اللِّقَاءِ}{再见}
\end{itemize}

\begin{Arabic}
    - أَهْلاََ وَسَهْلاََ! 

    - أَهْلاََ وَسَهْلاََ!

    - مَا اسْمُكَ؟

    - اِسْمِى أَهْمَدُ. وَمَا اسْمُكَ؟

    - اِسْمِى مَنْصُورٌ.

    - أَهْلاََ وَسَهْلاََ!

    - أَهْلاََ بِكَ!

    - أَنَا مَنْصُورٌ. وَمَا اسْمُكِ؟

    - اِسْمِى يُمْنَى.

    - حَسَنََا.إِلَى للِّقَاءِ!

    - إِلَى للِّقَاءِ! 
\end{Arabic}

\begin{note}
    \arm{مَا اسْمُكَ؟}读音为\arm{مَسْمُكَ}。

    \arm{إِلَى} (介词)后面 \arm{لِقَاءٌ} 变为属格 \arm{لِقَاءِ} 。
\end{note}

\section{语法}

\subsection{音节和重音}

短音节和长音节按直觉来。

\begin{itemize}
    \item 单音节词一律重读。
    \item 双音节词重读第一个音节。
    \item 多音节词:
    \begin{itemize}
        \item 词尾长音节不重读。
        \item 有叠音的重读叠音前。
        \item 两个以上长音节者,重读最后一个长音节。
        \item 重读倒数二三音节中的长音节,若有。
        \item 倒数第二个音节是短音节者,重读倒数第三个音节。
    \end{itemize}
\end{itemize}

\begin{note}
    似乎没有两个长音组成的双音节词?
\end{note}

\begin{note}
    \arm{ـََ} 和 \arm{ـَنْ} 的发音没有区别,但课中没有提到二者的区别,我猜测的区别有:
    \begin{itemize}
        \item 正词法规定;
        \item \arm{ـَنْ} 比 \arm{ـََ} 多一个辅音 \arm{ن} ,而闪语族语言的辅音可能有特殊含义。
    \end{itemize}
\end{note}

\subsection{名词的性}

名词分阴阳性,没有中性。大多数阴性名词有以下规律:

\begin{itemize}
    \item 绝大部分国名、城市名、部落名。
    \begin{itemize}
        \item \ac{دِمَشْقُ}{大马士革}
        \item \ac{بَكِينُ}{北京}
        \item \ac{سُورِيَا}{叙利亚}
        \item \ac{اَلصِّنُ}{中国}
    \end{itemize}
    \item 大部分成双的肢体和器官。
    \begin{itemize}
        \item \ac{قَدَمٌ}{脚}
        \item \ac{عَيْنٌص}{眼睛}
        \item \ac{كَفٌّ}{手掌}
        \item \ac{يَدٌ}{手}
    \end{itemize}
    \item 指物名词的复数当作阴性单数看待。
\end{itemize}

仅有有限的阴性名词没有规律:

\begin{itemize}
    \item \ac{نَارٌ}{火}
    \item \ac{شَمُسٌ}{日}
    \item \ac{أَرْضٌ}{地}
    \item \ac{سِنٌّ}{牙}
    \item \ac{دَارٌ}{房}
    \item \ac{حَرەبٌ}{战争}
    \item \ac{رِيحٌ}{风}
    \item \ac{رُوحٌ}{灵魂、精神}
    \item …………
\end{itemize}

\subsubsection{阴性标志}

绝大多数具有阴性标志的名词是阴性名词。

\begin{note}
    对于大多数情况:
    
    \begin{center}
        带有阴性标志 $\Rightarrow $ 阴性名词
    \end{center}

    且课上没有说反向是否可以。
\end{note}

阴性标志有如下几个。注意:全都有例外。

\paragraph{\arm{ة/ـة}} 

\begin{itemize}
    \item \ac{سَعَادَةٌ}{幸福}
    \item \ac{حَيَاةٌ}{生活}
    \item \ac{شرِكَةٌ}{公司}
\end{itemize}

可以在阳性名词后面加上 \arm{ة/ـة} 使其变成阴性。

\begin{itemize}
    \item \ac{فَتََى $\leftarrow $ فَتَاةٌ}{小伙 $\rightarrow $ 姑娘}
    \item \ac{جَمِيلٌ $\leftarrow $ جَمِيلَةٌ}{美丽的}
    \item \ac{لَيْلٌ $\leftarrow $ لَيْلَةٌ}{夜晚 $\rightarrow $ 一个夜晚}
    \item \ac{خُبْزٌ $\leftarrow $ خُبْزَةٌ}{面包 $\rightarrow $ 一块面包}
\end{itemize}

\paragraph{\arm{ـَى}}

\begin{itemize}
    \item \ac{بُشْرَى}{喜讯}
    \item \ac{ذِكْرَى}{纪念}
    \item \ac{كُبْرَى}{最大的(阴性)/la plus grande}
    \item \ac{يُمْنَى}{(人名)}
\end{itemize}

\paragraph{\arm{ـَاءُ}}

\begin{itemize}
    \item \ac{صَحْرَاءُ}{沙漠}
    \item \ac{سَوْدَاءُ}{黑色的(阴性)/noire}
\end{itemize}

\subsubsection{共性名词}

阴阳性通用的名词叫作共性名词。

\begin{itemize}
    \item \ac{صَبُورٌ}{忍耐的}
    \item \ac{قَتِيلٌ}{被杀的}
    \item \ac{سَمَاءٌ}{天空}
    \item \ac{سُوقٌ}{市场}
\end{itemize}

\subsection{名词的格}

属格作为介词受词。属格作为偏正组合偏次。

\paragraph{定尾名词} 无论什么格都固定尾符的名词。

\begin{itemize}
    \item \ac{مَنْ}{(疑问代词)谁}
    \item \ac{مَا}{(疑问代词)什么}
    \item \ac{هَذَا}{(指示代词)这个}
    \item \ac{يُمْنَى}{(人名)}
\end{itemize}

\begin{note}
    \arm{هَذَا} 注音为 \arm{هَاذَا} 。考虑到前面也对于同一个词出现过这个情况,应该是特殊读音。
\end{note}

\paragraph{变尾名词} 通常尾符随格变化的名词。

\begin{center}
    \begin{tabular}{c|ccc}
        & 主 & 宾 & 属 \\
        \hline
        确指 & \arm{ـُ} & \arm{ـَ} & \arm{ـِ}\\
        泛指 & \arm{ـٌ} & \arm{ـََا} & \arm{ـِِ}
    \end{tabular}
\end{center}

以 \arm{بَيْتٌ} (房子)为例。

\begin{center}
    \begin{tabular}{c|ccc}
        & 主 & 宾 & 属 \\
        \hline
        确指 & \arm{اَلْبَيْتُ} & \arm{اَلْبَيْتَ} & \arm{اَلْبَيْتِ}\\
        泛指 & \arm{بَيْتٌ} & \arm{بَيْتََا} & \arm{بَيْتِِ}
    \end{tabular}
\end{center}

对于一些人名地名,确指也标双音符。

\begin{center}
    \begin{tabular}{c|ccc}
        & 主 & 宾 & 属 \\
        \hline
        确指 & \arm{مَنْصُورٌ} & \arm{مَنْصُورََا} & \arm{مَنْصُورِِ}\\
    \end{tabular}
\end{center}

\paragraph{半变尾名词} 尾符随格而有部分变化的名词,如一些人名和地名。特点是词尾不标双音符。

\begin{center}
    \begin{tabular}{c|ccc}
        & 主 & 宾 & 属 \\
        \hline
        大马士革 & \arm{دِمَشْقُ} & \arm{دِمَشْقَ} & \arm{دِمَشْقَ}(不是 \arm{..ـقِ})\\
        艾哈迈德 & \arm{أَحْمَدُ} & \arm{أَحْمَدَ} & \arm{أَحْمَدَ}(不是 \arm{..ـدِ})
    \end{tabular}
\end{center}

但是,半变尾名词加冠词 \arm{اَلْ} 或后面被属格名词修饰(即作偏正组合正次),则属格为 \arm{ـِ}。