\chapter{墙外有两株树}

\section{\lecon{13.2} 单词和课文}

\begin{note}
    \lecon{13.2} 在网上大多资源没有声音,带声音的版本极难找,因此此课为抢救性质的笔记,因此较为详细。
\end{note}

\begin{itemize}
    \item \ac{سَلَامٌ}{和平}
    \item \ac{عَلَى}{在……上(后接属格接尾代名词时变成 \arm{عَلَيـ})}
    \item \ac{عَلَيكُمْ}{在您之上}
    \item \ac{السَّلَامُ عَلَيكُمْ}{您好,祝您平安}
    \item \ac{وَعَلَيْكُمُ السَّلَامُ}{也祝您平安( \arm{عَلَيكُمْ} 词尾 \arm{ـمْ} 遇后 \arm{ـْ} 需变动符,读作 \arm{ـمُ سْـ})}
    \item \ac{سَلَامَةٌ}{和平,平安}
    \item \ac{مَعَ السَّلَامَةِ}{再见(习惯上只用阴性)}
    \item \ac{حَمْدٌ}{赞颂}
    \item \ac{اَللّٰهُ}{真主,神}
    \item \ac{لِلّٰهِ}{属于真主}
    \item \ac{اَلْحَمْدُ لِلّٰهِ}{赞美真主}
    \item \ac{حَانَ}{v. (时间)到了}
    \item \ac{وَقْتٌ جـ أَوْقَاتَُ}{时间}
    \item \ac{مُحَاضَرَةٌ جـ محَاضَرَاتٌ}{课,讲座}
    \item \ac{اِسْتَأْذَنَ}{v. 告辞(第三人称单数阳性过去时)}
    \item \ac{أَسْتَأْذِنُ }{我告辞}
    \item \ac{مَعَ}{[虚] 表伴随}
    \item \ac{مَاذَا}{(有)什么}
    \item \ac{غُرْفَةٌ جـ غُرَفٌ}{房间}
    \item \ac{مَكْتَبٌ جـ مَكَاتِبُ}{课桌,办公室(bureau,复数半变尾)}
    \item \ac{سَرِييٌ جـ أَسِرَّةٌ}{床}
    \item \ac{مَقْعَرٌ جـ مَقَاعِدُ}{座椅(元音浑变)}
    \item \ac{قَامُوسٌ جـ قَوَمِيسٌ }{字典(复数半变尾)}
    \item \ac{جِدَارٌ جـ جُدُرٌ، أَجْدِرَةٌ}{墙壁(两种复数)}
    \item \ac{سَاعَةٌ جـ سَاعَاتٌ}{钟表,小时}
    \item \ac{خَرِيطَتٌ جـ خَرَائِطُ}{地图(复数半变尾)}
    \item \ac{لِحَافٌ جـ لُحُفٌ}{被子}
    \item \ac{مَخَدَّةٌ جـ مَخَادُّ}{枕头(〔西〕almohada,复数半变尾)}
    \item \ac{حَقِيبَةٌ جـ حَقَائِبُ}{包,书包(复数半变尾)}
    \item \ac{دَفْتَرٌ جـ دَفَاتِرُ}{本子(复数半变尾)} 
\end{itemize}

\begin{attention}
    \arm{اَللّٰهُ} 中带了一个冠词。此外,该词亦有独立编码 \arm{ﷲ} 。 \arm{ـلّـ} 上小竖表示发开口长音 \arm{ـلَّاـ} 。此处发音应浑厚。 
    
    \arm{لِلّٰهِ} 的开头实际上是介词 \arm{لِـ} ,发音不再浑厚。

    \arm{مَاذَا} 用于提问``有什么'',\arm{مَا} 用于提问``是什么''。 

    元音浑变,即 \arm{مَقْعَرٌ/مَقَاعِدُ} 中的 \arm{مَـ} 受后面 \arm{ـعَـ} 影响,从而发得更浑厚。

    \arm{مَخَادُّ} 中长音接静符号变短音,读作 \arm{مَخَ دُّ} 。
\end{attention}

\begin{Arabic}
    - السَّلَامُ عَلَيْكُمْ!

    - وَعَلَيْكُمُ السَّلَامُ!

    - كَيْفَ حَالُكِ؟

    - بِخَيْرٍ، اَلْحَمْدُ لِلّٰهِ. وَأَنْتَ؟

    - اَلْحَمْدُ لِلّٰهِ.

    - حَانَ وَقْتُ الْمُحَاضَرَةِ، أَسْتَأْذِنُكَ.

    - مَعَ السَّلَامَةِ!

    - إِلَى اللِّقَاءِ!
\end{Arabic}

\begin{attention}
    \arm{أَسْتَأْذِنُكَ} 带有宾格接尾代名词 \arm{ـكَ} ,表示``我向你告辞''。
\end{attention}

\begin{Arabic}
    - مَاذَا فِي الْغُرْفَةِ؟

    - فِي الْغُرْفَةِ مَكَاتِبٌ وَأَسِرَّةٌ وَمَقَاعِدُ.

    - وَمَاذَا عَلَى الْمَكْتَبِ؟

    - عَلَى الْمَكْتَبِ قَامُوسٌ.

    - وَمَاذَا عَلَى الْجِدَارِ؟

    - عَلَى الْجِدَارِ سَاعَةٌ وَخَرِيطَةٌ.

    - وَمَاذَا عَلَى السَّرِيرِ؟

    - عَلَى السَّرِيرِ لِحَافٌ وَمَخَدَّةٌ.

\end{Arabic}

\begin{attention}
    形如 \arm{فِي الْغُرْفَةِ} 中长音遇静符改短。
\end{attention}

\begin{Arabic}
    - وَمَاذَا عَلَى الْمَقْعَدِ؟

    - عَلَى الْمَقْعَدِ حَقِيبَةٌ.

    - وَمَاذَا فِيهَا؟

    - فِيهَا كُتُبٌ وَدَفَاتِرُ.
\end{Arabic}

\begin{attention}
    \arm{فِيهَا} 即 \arm{فِيـ} + \arm{ـهَا} ,表示``在她(属格第三人称阴性单数)里面''。如果是复数,依然当作阴性单数看待。
\end{attention}

\paragraph{\arm{حَانَ وَقْتُ...}} ……的时候到了:
\begin{itemize}
    \item \ac{حَانَ وَقْتُ الْمُحَاضَرَةِ.}{该上课了。}
    \item \ac{حَانَ وَقْتُ النَّوْمِ.}{该睡觉了。}
    \item \ac{حَانَ وَقْتُ الذَّهَابِ.}{该走了。}
\end{itemize}

\paragraph{介词/方位名词 + 事物A属格确指 + 事物B主格泛指} 表示A的方位有B:
\begin{itemize}
    \item \ac{عَلَى الْمَكْتَبِ قَامُوسٌ.}{课桌上有一本字典。}
    \item \ac{أَمَامَ البَيْتِ شَجَرَةٌ.}{房前有一棵树。}
\end{itemize}

\subsection{补充词汇}

\begin{itemize}
    \item \ac{جُمِيَةٌ جـ جُمًى}{洋娃娃}
    \item \ac{مِصْبَاحٌ جـ مَصَابِيحُ}{灯}
    \item \ac{مَاءٌ}{水}
    \item \ac{سَمَكٌ جـ أَسْمَاكٌ}{鱼(复数当作阴性单数看待)}
    \item \ac{جَبَلٌ جـ جِبَالٌ}{山(复数当作阴性单数看待)}
    \item \ac{نَحْنُ جـ أَنْهَارٌ}{河水/河(复数当作阴性单数看待,且 \arm{ـنْـ} 要显读)}
    \item \ac{كَبِيرٌ م كَبِيرَةٌ}{大的}
    \item \ac{صَغِيرٌ م صَغِرَةٌ}{小的}
    \item \ac{طَوِيلٌ م طَوِيلَةٌ}{长的}
    \item \ac{قَصِيرٌ م قَصِيرَةٌ}{短的}
    \item \ac{جَرِيدٌ م جَرِيدَةٌ}{新的( \arm{جَرِيدٌ جـ جُدُدٌ} )}
    \item \ac{قَدِيمٌ م قَدِيمَةٌ}{旧的}
    \item \ac{جَمِيلٌ م جَمِيلَةٌ}{美的}
    \item \ac{قَبِيحٌ م قَنِيحَةٌ}{丑的}
    \item \ac{مُفِيدٌ م مُفِيدَةٌ}{有用/有益的(注意首音节是 \arm{ـُ} )}
    \item \ac{تَافِهٌ م تَافِهَةٌ}{没用/无聊的(注意长音位置不一样)}
\end{itemize}

\begin{attention}
    \arm{م} 表示其后是其前的阴性形式。
\end{attention}

\begin{note}
    大多数形容词(主格、单数)的形式是 \arm{ـَ ـِيـ ـٌ م ـَ ـِيـ ـَةٌ}。如果用等宽字体,单词表会对得很整齐。
\end{note}

\section{语法}

\subsection{\lecon{13.3} 介词/方位名词 + 属格}

介词和方位名词词形不同,用法相同。

\begin{itemize}
    \item 介词不能单独使用,与受词组成介词短语。介词受词必须为属格。
    \item 方位名词后加名词属格,组成方位词组——一种正偏词组,在意义上,名词做方位的参照物。
    \begin{note}
    感觉说的就是形如``这个上面是桌子的''这种表达。
    \end{note}
\end{itemize}

\subsubsection{介词举例}

\paragraph{\arm{لَـ}} 属于(偶尔读 \arm{لِـ} )

\begin{itemize}
    \item \ac{لَهُ أُخْتٌ.}{他有一个姐妹。}
    \item \ac{لِي صَدِيقٌ.}{我有一个朋友。}
    \item \ac{لِيُمْنَى قَلَمٌ.}{ \arm{يُمْنَى} 有一支笔。}
\end{itemize}
\begin{attention}
    此类称为``倒装名词句'',省略``有''。可以理解为``一支笔属于 \arm{يُمْنَى} ''等。
\end{attention}

\paragraph{\arm{فِي}} 在……里面

\begin{itemize}
    \item \ac{فِي حَقِيبَتِي كُتُبٌ وَدَفَاتِرُ.}{我包里有书和本子。}
    \item \ac{فِي غُرْفَتِهِ سَرِيرٌ.}{他房间里有一张床。}
    \begin{attention}
        \arm{ﻏُﺮْﻓَﺔٌ} 变属格 \arm{ﻏُﺮْﻓَﺔِ} 以 \arm{ـِ} 结尾,根据第三人称音变,后接的 \arm{ـهُ} 变成 \arm{ـهِ}。
    \end{attention}
\end{itemize}

\paragraph{\arm{عَلَى}} 在……上面
    
\begin{attention}
    \arm{عَلَى} 接属格接尾代名词时读软音,变成 \arm{عَلَيـ} ,如 \arm{عَلَيهِ} (在他上面)。
\end{attention}

\begin{itemize}
    \item \ac{عَلَى الْمَكْتَبِ قَامُوسٌ.}{课桌上有一本字典。}
    \item \ac{عَلَى سَرِيرِهَا دُمْيَةٌ.}{她的床上有一个娃娃。}
\end{itemize}


\subsubsection{方位名词举例}

\paragraph{\arm{فَوقَ}} (悬/挂)于上方

\begin{itemize}
    \item \ac{فَوقَ الْمَكْتَبِ مِصْبَاحٌ.}{桌上悬有一盏灯。}
\end{itemize}

\begin{note}
    \begin{center}
        % \begin{center}
\begin{tikzpicture}[scale=0.5]
    \draw[thick] (-2,2) -- (2,2);
    \draw[thick] (-2,-2) -- (2,-2);

    % 桌子
    \draw (-1.5,-1) -- (1.5,-1); % 桌面
    \draw (-1,-1) -- (-1,-2); % 左腿
    \draw (1,-1) -- (1,-2);   % 右腿

    % 吊灯的线
    \draw (0,2) -- (0,1.5);

    % 灯泡
    \draw (0,1) circle (0.3);

    % 灯罩(倒三角形)
    \draw[fill=white] (-0.5,1) -- (0.5,1) -- (0,1.5) -- cycle;

    \node[below=5pt] at (current bounding box.south) {\arm{فَوقَ الْمَكْتَبِ مِصْبَاحٌ.}};
\end{tikzpicture}
% \end{center}
        % \begin{center}
\begin{tikzpicture}[scale=0.5]
    \draw[thick] (-2,2) -- (2,2);
    \draw[thick] (-2,-2) -- (2,-2);

    % 桌子
    \draw (-1.5,-1) -- (1.5,-1); % 桌面
    \draw (-1,-1) -- (-1,-2); % 左腿
    \draw (1,-1) -- (1,-2);   % 右腿

    \draw[rounded corners=5pt] (-0.5,-1) -- (-0.5,1) -- (0.5,1) --(0.5,0.8);

    % 灯泡
    \draw (0.5,0.3) circle (0.3);

    % 灯罩(倒三角形)
    \draw[fill=white] (0,0.3) -- (1,0.3) -- (0.5,0.8) -- cycle;

    \node[below=5pt] at (current bounding box.south) {\arm{عَلَى الْمَكْتَبِ مِصْبَاحٌ.}};
\end{tikzpicture}
% \end{center}
    \end{center}
\end{note}

\begin{attention}
    此外,需要注意 \arm{عَلَى} 和 \arm{فَوقَ} 的词性并不相同。
\end{attention}
    
\paragraph{\arm{تَحْتَ}} 在……下面。
\begin{itemize}
    \item \ac{تَحْتَ الْمَاءِ أَسْمَاكٌ.}{水下有鱼。}
\end{itemize}

\paragraph{\arm{أَمَامَ}} 在……前面。
\begin{itemize}
    \item \ac{أَمَامَ الْبَيْتِ شَجَرَةٌ.}{房前有一棵树。}
    \item \ac{أَمَامَ السَّرِيرِ مَكْتَبٌ.}{窗前有一张课桌。}
\end{itemize}

\paragraph{\arm{وَرَاءَ}} 在……后面。
\begin{itemize}
    \item \ac{وَرَاءَ الْبَيْتِ جَبَلٌ.}{房后有一座山。}
    \item \ac{وَرَاءَ الْمَكْتَبِ سَرِيرٌ.}{课桌后有一张床。}
\end{itemize}   

\paragraph{\arm{بَيْنَ}} 在……之间(同时指entre和parmi)。
\begin{itemize}
    \item \ac{بَيْنَ السَّرِيرِ وَالْمَكْتَبِ مَقْعَدٌ.}{(作entre)床和课桌之间有一把椅子。}
    \item \ac{بَيْنَ الْجِبَالِ نَهْرٌ.}{(作parmi)群山之间有一条河。}
\end{itemize}

\begin{note}
    \begin{center}
        % \begin{center}
\begin{tikzpicture}[
    scale=2,              % 图形缩放比例
    every node/.style={}, % 所有节点字体大小
    vertex/.style={circle, fill=black, inner sep=1.5pt}, % 顶点样式
    edge label/.style={midway, fill=white, inner sep=1pt}, % 边注释样式
]

\coordinate (O) at (0,0,0);
\coordinate (L) at (-1,0,0);
\coordinate (R) at (1,0,0);
\coordinate (U) at (0,1,0);
\coordinate (D) at (0,-1,0);
\coordinate (B) at (0,0,1);
\coordinate (F) at (0,0,-1);

\draw (D) node[label=right:{\arm{تَحْتَ}}] {};
\draw (F) node[label=above right:{\arm{أَمَامَ}}] {};
\draw (U) node[label=left:{\arm{فَوقَ}}] {};
\draw (B) node[label=below left:{\arm{وَرَاءَ}}] {};
\draw (O) node[vertex, label=above left:{\arm{بَيْنَ}}] {};

\draw[-{Stealth[length=3mm]}, dashed] (O) -- (L) node[edge label]{\footnotesize 左};
\draw[] (O) -- (D) node[edge label]{\footnotesize 下};
\draw[-{Stealth[length=3mm]}] (O) -- (F) node[edge label, sloped]{\footnotesize 前};
\draw[dashed] (O) -- (R) node[edge label]{\footnotesize 右};
\draw[-{Stealth[length=3mm]}] (O) -- (U) node[edge label]{\footnotesize 上};
\draw[] (O) -- (B) node[edge label, sloped]{\footnotesize 后};

\node[below=5pt] at (current bounding box.south) {方位名词};

\end{tikzpicture}
% \end{center}
    \end{center}
\end{note}

\subsection{\lecon{13.4} 定语和被修饰语}

定语可以是形容词(属于名词)/词组/从句。定语后置,性、数、\emph{格、式}均要配合。

\begin{note}
    所谓``式'',就是泛指/确指(不叫特指)。
\end{note}

以下是性数格式配合的若干例子。

\begin{Arabic}
    \paragraph{كِتَابٌ مُفِيدٌ}
    \begin{description}
        \item [م] خَرِيطَةٌ مُفِيدَةٌ
    \end{description}

    \paragraph{طَالِبٌ جَدِيدٌ}
    \begin{description}
        \item [م] طَالِبَةٌ جَدِيدَةٌ
        \item [جـ] طَلَبَةٌ جُدُرٌ
    \end{description}

    \paragraph{كِتَابٌ قَرِيمٌ}
    \begin{description}
        \item [جـ] كُتُبٌ قَرِيمَةٌ
    \end{description}

    \paragraph{زَمِيلٌ قَدِيمٌ}
    \begin{description}
        \item [宾] زَمِيلََا قَدِيمََا
        \item [属] زَمِيلِِ قَدِيمِِ
        \item [确] زَمِيلِي الْقَدِيمُ
    \end{description}

    \paragraph{فَتَاةٌ جَمِيلَةٌ}
    \begin{description}
        \item [宾] فَتَاةً جَمِيلَةً
        \item [属] فَتَاةٍ جَمِيلَةٍ
        \item [确] اَلْفَتَاةُ الْجَمِيلَةُ 
    \end{description}
\end{Arabic}


\begin{attention}
    记得指物名词复数当作阴性单数看待。

    \arm{زَمِيلِي الْقَدِيمُ} 一例中,使用属格接尾代名词将 \arm{زَمِيلٌ} 变成了确知名词。
\end{attention}

\begin{note}
    记不住的词去前面扩展词汇查。

    记得 \arm{ة + ـََا} 是 \arm{ةََ} ,没有 \arm{ا}。
\end{note}