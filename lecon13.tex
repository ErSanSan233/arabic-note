\chapter{描述物品2}

\section{\lecon{13.2}单词和课文}

\begin{itemize}
    \item \ac{سَلَامٌ}{和平}
    \item \ac{عَلَى}{在……上(后接属格接尾代名词时变成 \arm{عَلَيـ})}
    \item \ac{عَلَيكُمْ}{在您之上}
    \item \ac{السَّلَامُ عَلَيكُمْ}{您好,祝您平安}
    \item \ac{وَعَلَيْكُمُ السَّلَامُ}{也祝您平安( \arm{عَلَيكُمْ} 词尾 \arm{ـمْ} 遇后 \arm{ـْ} 需变动符,读作 \arm{ـمُ سْـ})}
    \item \ac{سَلَامَةٌ}{和平,平安}
    \item \ac{مَعَ السَّلَامَةِ}{再见(习惯上只用阴性)}
    \item \ac{حَمْدٌ}{赞颂}
    \item \ac{اَللّٰهُ}{真主,神}
    \item \ac{لِلّٰهِ}{属于真主}
    \item \ac{اَلْحَمْدُ لِلّٰهِ}{赞美真主}
    \item \ac{حَانَ}{v. (时间)到了}
    \item \ac{وَقْتٌ جـ أَوْقَاتَُ}{时间}
    \item \ac{مُحَاضَرَةٌ جـ محَاضَرَاتٌ}{课,讲座}
    \item \ac{اِسْتَأْذَنَ}{v. 告辞(第三人称单数阳性过去时)}
    \item \ac{أَسْتَأْذِنُ }{我告辞}
    \item \ac{مَعَ}{[虚] 表伴随}
    \item \ac{مَاذَا}{(有)什么}
    \item \ac{غُرْفَةٌ جـ غُرَفٌ}{房间}
    \item \ac{مَكْتَبٌ جـ مَكَاتِبُ}{课桌,办公室(bureau,复数半变尾)}
    \item \ac{سَرِييٌ جـ أَسِرَّةٌ}{床}
    \item \ac{مَقْعَرٌ جـ مَقَاعِدُ}{座椅(元音浑变)}
    \item \ac{قَامُوسٌ جـ قَوَمِيسٌ }{字典(复数半变尾)}
    \item \ac{جِدَارٌ جـ جُدُرٌ، أَجْدِرَةٌ}{墙壁(两种复数)}
    \item \ac{سَاعَةٌ جـ سَاعَاتٌ}{钟表,小时}
    \item \ac{خَرِيطَتٌ جـ خَرَائِطُ}{地图(复数半变尾)}
    \item \ac{لِحَافٌ جـ لُحُفٌ}{被子}
    \item \ac{مَخَدَّةٌ جـ مَخَادُّ}{枕头(〔西〕almohada,复数半变尾)}
    \item \ac{حَقِيبَةٌ جـ حَقَائِبُ}{包,书包(复数半变尾)}
    \item \ac{دَفْتَرٌ جـ دَفَاتِرُ}{本子(复数半变尾)} 
\end{itemize}

\begin{note}
    \arm{اَللّٰهُ} 中带了一个冠词。此外,该词亦有独立编码 \arm{ﷲ} 。 \arm{ـلّـ} 上小竖表示发开口长音 \arm{ـلَّاـ} 。此处发音应浑厚。 
    
    \arm{لِلّٰهِ} 的开头实际上是介词 \arm{لِـ} ,发音不再浑厚。

    \arm{مَاذَا} 用于提问``有什么'',\arm{مَا} 用于提问``是什么''。 

    元音浑变,即 \arm{مَقْعَرٌ/مَقَاعِدُ} 中的 \arm{مَـ} 受后面 \arm{ـعَـ} 影响,从而发得更浑厚。

    \arm{مَخَادُّ} 中长音接静符号变短音,读作 \arm{مَخَ دُّ} 。
\end{note}

\begin{Arabic}
    - السَّلَامُ عَلَيْكُمْ!

    - وَعَلَيْكُمُ السَّلَامُ!

    - كَيْفَ حَالُكِ؟


\end{Arabic}