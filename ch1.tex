\chapter{\lecon{1} ~ \lecon{7}引言和字母表部分}

\section{字母表(\arm{اَلْأَبْجَدِيَّةُ الْعَرَبِيَّةُ})及特殊字母写法}

\begin{note}
    字母表请务必跟着视频课学习。笔记只能呈现字母的部分形态。

    此外,在学习阿语字母表时,我愈发意识到,对于老师来说,阿语的一些字形可能已经很熟悉,因此讲解时会快速带过,或干脆略过。但我们来说,面对这样一款陌生的语言,我们需要尝试用最快的速度掌握分辨字形的合适粒度。也就是说,我们需要知道对于这套文字,哪些地方是装饰、哪些地方是区别字母的关键、哪些地方是字体的差异。

    举个例子,\arm{ل} 与 \arm{ا} 搭配而成的 \arm{لا} 有一个独特的词尾型 \arm{ـلا}。在字形呈现上, \arm{لا} 和 \arm{ـلا} 可能很相似,也可能相差很大。这是视频课中没有明确提到的,需要我们自己去观察发现。
\end{note}

\begin{center}
\begin{Arabic}
    \begin{longtable}{c|r|r|r|r}
        \crm{独立型} & \crm{读音} & \crm{词头、中、尾连写} & \crm{手写} & \crm{特殊形态} \\
        \hline
        أ & أَلِفٌ & أـأـأ & \emph{أـأـأ} & إِ آ \\
        ب & بَاءٌ & بـبـب & \emph{ببب}\\
        ت & تَاءٌ & تـتـت & \emph{تتت} & ة/ـة ةً \\
        ث & ثَاءٌ & ثـثـث & \emph{ثثث}\\
        ج & جِيمٌ & جـجـج & \emph{ججج}\\
        ح & حَاءٌ & حـحـح & \emph{ححح}\\
        خ & خَاءٌ (厚) & خـخـخ & \emph{خخخ}\\
        د & دَالٌ & دـدـد & \emph{دـدـد}\\
        ذ & ذَالٌ (咬) & ذـذـذ & \emph{ذـذـذ}\\
        ر & رَاءٌ (厚) & رـرـر & \emph{رـرـر}\\
        ز & زَاىٌ & زـزـز & \emph{زـزـز}\\
        س & سِينٌ & سـسـس & \emph{سسس}\\
        ش & شِينٌ & شـشـش & \emph{ششش}\\
        ص & صَادٌ (厚) & صـصـص & \emph{صصص}\\
        ض & ضَادٌ (厚) & ضـضـض & \emph{ضضض}\\
        ط & طَاءٌ (厚) & طـطـط & \emph{طـطـط}\\
        ظ & ظَاءٌ (厚/咬) & ظـظـظ & \emph{ظـظـظ}\\
        ع & عَينٌ & عـعـع & \emph{ععع}\\
        غ & غَينٌ & غـغـغ & \emph{غغغ}\\
        ف & فَاءٌ & فـفـف & \emph{ففف}\\
        ق & قَافٌ (厚) & قـقـق & \emph{ققق}\\
        ك & كَافٌ & كـكـك & \emph{ككك}\\
        ل & لَامٌ & لـلـل & \emph{للل} & لا/ـلا لاَ لاً \\
        م & مِيمٌ & مـمـم & \emph{ممم}\\
        ن & نُونٌ & نـنـن & \emph{ننن}\\
        ه & هَاءٌ & هـهـه & \emph{ههه}\\
        و & وَاوٌ & وـوـو & \emph{وـوـو}\\
        ى & يَاءٌ & يـيـى & \emph{ييى}\\
    \end{longtable}
\end{Arabic}
\end{center}

\begin{note}
    有资料表示,课上讲的手写体是卢格阿体,但我目前还没有办法分辨几种手写阿语字体的区别,所以只是找了看起来好看的字库文件。
    尤其是,我很想找到一个把 \arm{سسسـ} 完全渲染成一条横线,并且 \arm{ــ} 能够有一定长度的字库。很可惜,目前还没找到特别合适的。
\end{note}

\begin{itemize}
    \item 标``厚''表示其后加 \arm{ـَ} 或 \arm{ـَا} 时发音要更浑厚(靠后,类似\textipa{[A]})。
    \item 标``咬''表示咬舌发音。
    \item \arm{ة} 常位于词尾,一般为阴性名词标志,其前字母永远标 \arm{ـَ} 或 \arm{ـَا} 。\arm{ة} 搭配 \arm{ــًا} 时不写 \arm{ا}。
    \item \ac{ضادَ}{打猎}
    \item \ac{عَيْنٌ}{眼睛} 
    \item \ac{قِرْدٌ}{猴子}
    \item \ac{كَفٌّ}{手}
    \item \ac{مَاءٌ}{水}
\end{itemize}

以下是字母学习中的拼写示例,截至整理此表时已经整理过的单词不再重复列出:

\begin{itemize}
    \item \lecon{2} \ac{أَ}{[虚](难道)……吗?}
    \item \ac{أَوْ}{[虚]或者}
    \item \ac{آهْ}{啊}
    \item \ac{آوَى}{给予庇护}
    \item \ac{وَأْوَأِ}{v. 狗叫}

    \item \lecon{3} \ac{بَابٌ}{门}
    \item \ac{أَبٌ}{父亲}
    \item \ac{تُوتٌ}{桑葚}
    \item \ac{أَتَى}{来到}
    \item \ac{أَتَيتُ}{我来了}
    \item \ac{أَثَاثٌ}{家具}
    \item \ac{ثَوبٌ}{衣服}
    \item \ac{أَثْبَتَ}{v. 证明}
    \item \ac{هَيْئَةٌ}{机构}

    \item \lecon{4} \ac{جَاءَ}{到来}
    \item \ac{وَجْهٌ}{脸}
    \item \ac{وَاحِدٌ}{一}
    \item \ac{حَوْتٌ}{鲸}
    \item \ac{أَخٌ}{兄弟}
    \item \ac{أُخْتٌ}{姐妹}
    \item \ac{أَبَدٌ}{n. 永远}
    \item \ac{دَجَاجٌ}{母鸡}
    \item \ac{أَذًى}{n. 伤害}
    \item \ac{ذَهَبَ}{v. 去}
    \item \ac{رَأَى}{看}
    \item \ac{وَرَاءَ}{在……后面}
    \item \ac{زُبْدَةٌ}{黄油}
    \item \ac{زَبِيبٌ}{葡萄干}
    \item \ac{حَرْبٌ}{战争}
    \item \ac{حَزْبٌ}{党派}
    
    \item \lecon{5} \ac{سَحَابٌ}{云}
    \item \ac{أَسْوٌدُ}{n. 黑色的}
    \item \ac{سُؤَالٌ}{问题}
    \item \ac{شَجَرٌ}{树}
    \item \ac{شَبْخٌ}{老人}
    \item \ac{بَشَرٌ}{人类}
    \item \ac{صَوبٌ}{声音}
    \item \ac{صَائِدٌ}{猎手}
    \item \ac{صُورَةٌ}{照片}
    \item \ac{ضَرَبَ}{打}
    \item \ac{طَبِيبٌ}{医生}
    \item \ac{ضَابِطٌ}{军官}
    \item \ac{اَلْخِطَا}{契丹人}
    \item \ac{ظَهَرَ}{显现}
    \item \ac{ظَبْيٌ}{羚羊}
    \item \ac{أَبُوظَبِي}{阿布扎比}
    \item \ac{عَاشَ}{v. 生活}
    \item \ac{سَاعَةٌ}{小时、钟表}
    \item \ac{ضَبْعٌ}{鬣狗}
    \item \ac{غَرْغَرَ}{v. 漱口}
    \item \ac{بُلْغَارِيَا}{保加利亚}
    \item \ac{بَلَاغَةٌ}{口才}
    
    \item \lecon{6} \ac{فَتًى}{小伙}
    \item \ac{فَتَاةٌ}{姑娘}
    \item \ac{فِي}{[虚]在……里面}
    \item \ac{فَوْقَ}{n. 在……上面}
    \item \ac{وَقَفَ}{站立、停止}
    \item \ac{وَافَقَ}{v. 同意}
    \item \ac{كُرَةٌ}{球}
    \item \ac{تُرْكِيَا}{土耳其}
    \item \ac{شَرِكَةٌ}{公司}
    \item \ac{لَوُ}{如果……}
    \item \ac{عَلَى}{[虚]在……上面}
    \item \ac{أَكَلَ}{吃}
    \item \ac{مِنْ}{从}
    \item \ac{قَلَمٌ}{笔}
    \item \ac{مَنَعَ}{v. 禁止}
    \item \ac{اَلْيَمَنُ}{也门}
\end{itemize}

\begin{note}
    我觉得,如果过多按照独立型去背诵字母形态,在识读单词的时候会遇到更大的障碍,因为单词中的单词大多数处于词中型。既然如此,不妨按照词中型的形态将字母整理出来呢。

    我根据自己的理解,依照课上的手写体,给每个形状取了名字。此外,请注意阅读顺序(从两边到中间)。
    \begin{itemize}
        \item \ac{ـهـ و ـمـ}{(不规则)h, w, m}
        \item \ac{ـئـ / ـنـ ـتـ ـثـ / ـبـ ـيـ}{(短牙型)' / n, t, th / b, y}
        \item \ac{ـلـ ـكـ}{(长牙型)l, k}
        \item \ac{ـــحـ / ـــخـ / ـــجـ}{(闪电型)ḥ / kh / \textipa{\textdyoghlig}}
        \item \ac{د ذ / ر ز}{(小钩型)d, dh / r, z}
        \item \ac{ـسـ ـشـ}{(横线型)s, sh}
        \item \ac{ـفـ ـقـ / ـطـ ـظـ / ـصـ ـضـ}{(上圈型)f, q / ṭ, ẓ / ṣ, ḍ}
        \item \ac{ـعـ ـغـ}{(三角型)`, gh}
    \end{itemize}

    可以看到,转写中为拉丁字母加点,对应到阿文很可能会完全变成另一个字母。于是,我又整理了部分字母的发音关系。该表中,关于中心轴对称的辅音互为清浊关系。越远离中心轴,表示发音越难(加喉音、加顶音等)。对应转写并列展示。

    \begin{center}
        \begin{multicols}{2}
            \begin{tabular}{cc||cc}
                \hline
                \arm{ض} & \arm{د} & \arm{ت} & \arm{ط} \\
                && \arm{ك} & \arm{ق} \\
                & \arm{غ} & \arm{خ} \\
                \arm{ع} && \arm{ه} & \arm{ح}\\
                & \arm{ج} & \arm{ش} \\
                & \arm{ز} & \arm{س} & \arm{ص} \\
                \arm{ظ} & \arm{ذ} & \arm{ث} \\
                \hline
            \end{tabular}

            \begin{tabular}{cc||cc}
                \hline
                ḍ & d & t & ṭ \\
                && k & q \\
                & gh & kh \\
                ` && h & ḥ \\
                & \textipa{\textdyoghlig} & sh \\
                & z & s & ṣ \\
                ẓ & dh & th \\
                \hline
            \end{tabular}
        \end{multicols}
    \end{center}

    【后补】我发现 \arm{ظ} 和 \arm{ض} 比较难反应过来,可以这么记:它们两个好像反了, \arm{ظ} 像 \arm{ص} 的浊音,而 \arm{ض} 像 \arm{ط} 的浊音。
\end{note}

\section{单词和课文}

\begin{itemize}
    \item \ac{اَلْعَرَبِيَّةُ}{阿拉伯语}
    \item \ac{اَلْبُلْدَانُ الْعَرَبِيَّةُ}{阿拉伯国家}
    \item \ac{صَحْرَاءُ}{沙漠}
    \item \ac{اَصَّحْرَاءُ الْكُبْرَى}{撒哈拉沙漠(沙漠--最大的)}
    \item \ac{جَمَلٌ}{骆驼}
    \item \ac{نِفْطٌ}{石油}
    \item \ac{بِتْرُولٌ}{石油(音译自petrol)}
    \item \ac{قَهْوَةٌ}{咖啡}
    \item \ac{أَلْفُ لَيْلَةٍ وَلَيْلَةٌ}{《一千零一夜》(一千--夜(属)--和一夜)}
    \begin{itemize}
        \item (\lecon{6.4})\ac{أَلْفُ}{一千}
        \item \ac{لَيْلٌ}{夜晚(表类)}
        \item \ac{لَيْلَةٌ}{一个夜晚}
    \end{itemize}
    \item \ac{جُبْرَانُ}{纪伯伦}
    \item \ac{اَللُّغَاتُ السَّامِيَّةُ}{闪米特语族(Semitic Languages)}
\end{itemize}

\section{语法}

阿拉伯语属于闪含语系闪语族(闪米特语族),诞生于阿拉伯半岛,最早的文献可以追溯到公元6世纪阿拉伯半岛北部的石刻。

阿拉伯语辅音相对复杂,一般三个辅音字母构成基本含义。例如:

\begin{itemize}
    \item \ac{فَهِمَ}{理解}
    \item \ac{أَفِهَمَ}{使……理解}
    \item \ac{تَفَاهَمَ}{相互理解}
    \item \ac{اِسْتَفْهَمَ}{询问(要求理解)}
    \item \ac{مَفْهُومٌ}{概念(被理解的事物)}
\end{itemize}

阿拉伯语有名词、动词、虚词,没有系动词:

\begin{itemize}
    \item \ac{هَذَا كِتَابٌ.}{这--一本书。( \arm{هَذَا} 注音为\arm{هَاذَا})}
\end{itemize}

阿语动词常在句首:遇到--张三(主格)--李四(宾格)
